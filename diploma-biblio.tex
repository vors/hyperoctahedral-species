%\nocite{*} % Show all Bib-entries
\begin{thebibliography}{9}

\bibitem{J1} André Joyal, \emph{Une théorie combinatoire des séries
formelles}, Adv. Math 42 (1981), 1–82.

\bibitem{J2} André Joyal, \emph{Foncteurs analytiques et espèces des
structures}, in Combinatoire Énumérative, Lecture Notes in Mathematics 1234, Springer,
Berlin, (1986), pp. 126–159

\bibitem{BergTrees} F. Bergeron, Gilbert Labelle, Pierre LeRoux
\emph{Combinatorial Species and Tree-Like Structures}, Cambridge University Press, (1998)

\bibitem{BergH} N. Bergeron; P. Choquette.
\emph{Hyperoctahedral species}, Sém. Lothar. Combin. 61A (2009/10), доступно на
\url{http://arxiv.org/abs/0810.4089}

\bibitem{Cubic} Hetyei, Gábor; Labelle, Gilbert;
Leroux, Pierre \emph{Cubical species and nonassociative algebras} Adv. in Appl.
Math. (1998), no. 3

\bibitem{Mac1} I. G. Macdonald. \emph{Polynomial functors and wreath
products}, J. Pure Appl. Algebra, 18(2):173–204, 1980.

\bibitem{Mac2} I. G. Macdonald. \emph{Symmetric functions and Hall polynomials},
Oxford Mathemati- cal Monographs.
The Clarendon Press Oxford University Press, New York, second edition, 1995.

\bibitem{Day} \url{http://nlab.mathforge.org/nlab/show/Day+convolution}

\bibitem{Durov} Н. В. Дуров. \emph{Классифицирующие вектоиды и классы операд},
Санкт-Петербургское отделение Математического института им. В. А. Стеклова РАН,
Санкт-Петербург, Россия 2009, доступно (английский) на
\url{http://arxiv.org/abs/1105.3114}

\end{thebibliography}