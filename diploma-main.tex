% $Id: diploma.tex 90 2009-06-11 12:51:46Z zyv $

% The extsizes classes and class options provide support for sizes
% eight, nine,
% ten, eleven, twelve, fourteen, seventeen and twenty points.
% http://www.ctan.org/tex-archive/macros/ ... /extsizes/
\documentclass[a4paper,14pt]{extarticle}

% PDF search & cut'n'paste
\usepackage{cmap}

% Cyrillic support
\usepackage{mathtext}
\usepackage[T2A]{fontenc}
\DeclareSymbolFont{T2Aletters}{T2A}{cmr}{m}{it}
\usepackage[utf8]{inputenc}

% Also be sure to add
%
% english hyphen
% ruseng ruenhyph
% =russian
%
% to ~/.texlive2008/texmf-var/tex/generic/config/language.dat

\usepackage[english,russian]{babel}

% AMS font faces
\usepackage{amsmath, amsfonts, amssymb}


% Support for the upright and bold greek letters
\usepackage{bm}
\usepackage[Symbolsmallscale]{upgreek}
\makeatletter
        \newcommand{\bfgreek}[1]{\bm{\@nameuse{up#1}}}
\makeatother

% Detect whether PDFLaTeX is in use
\usepackage{ifpdf}

% Graphics
\ifpdf
        \usepackage[pdftex]{graphicx}
\else
        \usepackage{graphicx}
\fi
\graphicspath{{images/}}

% Tikz package for draw commutative diagrams (include after graphics)
\usepackage{tikz}
\usetikzlibrary{matrix,arrows}
\tikzset{node distance=2cm, auto}

% Indent the first paragraph as well
\usepackage{indentfirst}

% According to GOST, sections should be called chapters in diploma
\usepackage{titlesec}

\setcounter{tocdepth}{2}

\titleformat{\section}[block]{\bfseries\large\sffamily\raggedright}
        {Глава~\Roman{section}.}{1ex}{}
\titleformat{\subsection}[block]{\bfseries\normalsize\sffamily\raggedright}
        {\arabic{section}.\arabic{subsection}.}{1ex}{}
\titleformat{\subsubsection}[block]{\normalsize\sffamily\raggedright}
        {\arabic{section}.\arabic{subsection}.\arabic{subsubsection}.}{1ex}{}

\titlespacing*{\section}      {0pt}{3.50ex plus 1ex minus .2ex}{2.3ex
plus .2ex}
\titlespacing*{\subsection}   {0pt}{3.25ex plus 1ex minus .2ex}{1.5ex
plus .2ex}
\titlespacing*{\subsubsection}{0pt}{3.25ex plus 1ex minus .2ex}{1.5ex
plus .2ex}

\usepackage[titles]{tocloft}

\renewcommand{\cftsecpresnum}{Глава~}
\renewcommand{\cftsecleader}{\bfseries\cftdotfill{\cftdotsep}}
\renewcommand{\cftsecaftersnum}{.}
\renewcommand{\cftsubsecaftersnum}{.}

\newlength{\zyvseclen}
\settowidth{\zyvseclen}{\bfseries\cftsecpresnum\cftsecaftersnum}
\addtolength{\zyvseclen}{2mm}
\addtolength{\cftsecnumwidth}{\zyvseclen}

\renewcommand{\thesection}{\Roman{section}}
\renewcommand{\thesubsection}{\arabic{section}.\arabic{subsection}}
\renewcommand{\thesubsubsection}
        {\arabic{section}.\arabic{subsection}.\arabic{subsubsection}}

% Page numbering at the right topmost part of the page
\pagestyle{myheadings}

% Provides support for setting the spacing between lines in a document.
% Package
% options include singlespacing, onehalfspacing, and doublespacing.
% http://www.ctan.org/tex-archive/macros/ ... /setspace/
\usepackage{setspace}

% Alternative geometry
\usepackage[top=2cm, bottom=2cm, left=3cm, right=1cm]{geometry}

% Hyperlinks
\ifpdf
        \usepackage[pdftex]{hyperref}
\else
        \usepackage{hyperref}
\fi

\hypersetup{
        unicode=true,
        pdftitle={
        },
        pdfauthor={},
        pdfkeywords={
        },
        colorlinks,
        citecolor=black,
        filecolor=black,
        linkcolor=black,
        urlcolor=blue
}

% Fix links to floats
\usepackage[all]{hypcap}

% Nice citations [1,2,3,4] -> [1-4]
\usepackage[numbers,sort&compress]{natbib}

% [1] -> 1. in the bibliography
\makeatletter
\renewcommand\@biblabel[1]{#1.}
\makeatother

% Russian-styled figure and table captions
\usepackage[labelsep=period]{caption}

% Here we define the relationships for the counters: normaly we should
% reset the eq, figure & table counters every chapter
\makeatletter
\@addtoreset{equation}{section} % Equation counter
\@addtoreset{figure}{section} % Figure counter
\@addtoreset{table}{section} % Table counter
\makeatother

\renewcommand{\theequation}{\arabic{section}.\arabic{equation}}
\renewcommand{\thefigure}{\arabic{section}.\arabic{figure}}
\renewcommand{\thetable}{\arabic{section}.\arabic{table}}

% Keeps floats `in their place', preventing them from floating past a
% "\FloatBarrier" command into another section.  The floats should not
% move
% past every "\section".
\usepackage[section]{placeins}

% Compressed lists: compactitem etc.
\usepackage{paralist}

% Useful for individually placing figures on a separate page with
% \afterpage{\clearpage \begin{figure}[p] ... }
\usepackage{afterpage}

% Allow landscape pages for graphics, call like:
%
%       \afterpage{\clearpage
%       \begin{landscape}
%       \begin{figure}[p]
%       ...
%       \end{figure}
%       \end{landscape}
%       }
\ifpdf
        \usepackage{pdflscape}
\else
        \usepackage{lscape}
\fi

% This declaration makes TeX less fussy about line breaking. This can
% prevent overfull boxes, but may leave too much space between words.
% As this really isn't a fine art typography, we'll turn it on, so
% we won't have paragraphs which spans on the margins...
\sloppy

\begin{document}

% ------------------------------------------------------------------------------
% Обложка
% $Id: diploma-title.tex 52 2009-06-04 09:46:14Z zyv $

\thispagestyle{empty}

\begin{center}

        \textsc{Федеральное агентство по образованию}\\[0.2cm]

        \textsc{Государственное высшее учебное заведение \\
        <<Санкт-Петербургский государственный университет>>}\\[0.7cm]

        Математико--механический факультет \\[0.5cm]

        Специальность <<математика>>\\[0.7cm]

        Кафедра высшей алгебры и теории чисел\\[0.7cm]

        \textsc{Дипломная работа}\\[0.7cm]

        \begin{large}
                \textsc{\textbf{Гипероктаэдральные комбинаторные типы}}
        \end{large}

\end{center}

\vspace{0.7cm}

\textit{<<К защите допущен>>:}

\begin{center}
        \begin{tabular}{ll}
                Зав. кафедрой высшей алгебры \\ 
                и теории чисел, \\
                профессор, д.ф.-м.н. &
                        \begin{tabular}{ll}
                                \underline{\phantom{Четкая подпись}} &
                                Яковлев А.В.
                        \end{tabular}
        \\[0.7cm]
                Научный руководитель, \\
                доцент??, д.ф.-м.н. &
                        \begin{tabular}{ll}
                                \underline{\phantom{Четкая подпись}} &
                                Пименов К.И.
                        \end{tabular}
        \\[0.7cm]
                Рецензент, \\
                д.ф.-м.н. &
                        \begin{tabular}{ll}
                                \underline{\phantom{Четкая подпись}} &
                                ??? ?.?.
                        \end{tabular}
        \\[0.7cm]
        Дипломник &
                        \begin{tabular}{ll}
                                \underline{\phantom{Четкая подпись}} &
                                Ватманн В.В.
                        \end{tabular}
        \end{tabular}

        \vspace{1.5cm}

        г. Санкт-Петербург, 2012

\end{center}
 


% ------------------------------------------------------------------------------
% Оглавление
\tableofcontents

%-------------------------------------------------------------------------------
% New commands
\newcommand{\Zfull}{[\mathcal Z^{(1)} + 2\mathcal Z^{(2)}]}

%-------------------------------------------------------------------------------
% New draw commands
\newcommand{\dA}{
\begin{tikzpicture}
\draw (0pt,2pt) -- (0pt,6pt);
\draw (0pt,0pt) circle (2pt);
\draw (0pt,8pt) circle (2pt);
\end{tikzpicture}
} 

\newcommand{\dB}{
\begin{tikzpicture}
\draw (0pt,2pt) -- (0pt,6pt);
\draw[fill] (0pt,0pt) circle (2pt);
\draw (0pt,8pt) circle (2pt);
\end{tikzpicture}
} 

\newcommand{\dAA}{
\begin{tikzpicture}
\draw (0pt,0pt) -- (0pt,8pt);
\draw (0pt,0pt) -- (8pt,0pt);
\draw (8pt,8pt) -- (0pt,8pt);
\draw (8pt,8pt) -- (8pt,0pt);
\end{tikzpicture}
} 

\newcommand{\dAB}{
\begin{tikzpicture}
\draw[line width=1.5pt] (0pt,0pt) -- (0pt,8pt);
\draw[line width=1.5pt] (0pt,0pt) -- (8pt,0pt);
\draw (8pt,8pt) -- (0pt,8pt);
\draw (8pt,8pt) -- (8pt,0pt);
\end{tikzpicture}
} 

% ------------------------------------------------------------------------------
% Текст работы
\begin{onehalfspacing}
        \section{Введение}

\subsection{Комбинаторные виды}
Комбинаторные виды (\emph{species}) были введены Жуаялем в 1980 году \cite{J1}.
Они дают универсальный аппарат изучения помеченных (labeled) и
непомеченных (unlabeled) структур, и являются развитием идеи производящих
функций. О комбинаторных видах можно говорить на нескольких языках: категорном,
комбинаторном и на языке теории представлений. Последний наиболее часто
встречается в литературе, хотя автору он кажется наимение выразительным.
Во введении изложено начало теории комбинаторных видов. Основным источником
информации про комбнаторные виды является \cite{BergTrees}.
\subsubsection{Определение}
Рассмотрим категорию $\B$ --- группоид конечных множеств. Она эквивалентна
группоиду, объекты которого пронумерованы неотрицательными целыми числами и
$Hom(n, n) = S_n$.
\begin{definition}
Комбинаторным видом (species) называется функтор $$F:\B \rightarrow \Set$$
\end{definition}

Задать такой функтор, это то же самое что для каждого $n \in
\mathbb N$ задать множество $F[n]$ с действием группы $S_n$. В комбинаторике
такая ситуация возникает, когда мы рассматриваем явно определнные каким-либо
образом структуры на конечных множествах. Например: линейные порядки,
циклические порядки, деревья. Действие $S_n$ ествественно возникает из
перестановок исходных точек.
\begin{example}
Вид $\E$ --- вид множесто (без дополнительной структуры). Он
сопоставляет набору точек одно множество, состоящие из этих точек, 
$\E[n] =\{*\}$. Все элементы $S_n$ переходят в тождественное отображение. 
\end{example}
\begin{example}
$\mathbf C$ --- циклический порядок. Сопоставляет набору из $n$ точек
$(n-1)!$ возможных циклических порядков на них. 
\end{example}
\begin{example}
Линейный порядок $\mathbf L$ сопоставляет $n!$ линейных
порядков. 
\end{example}
\begin{example}
$\E_e$ --- сужение $\E$ на четные множества. То есть для четных
$n$, совпадает с $\E$, а для нечетных $\emptyset$. 
Аналогично $\E_o$ --- сужение на нечетные.
\end{example}
\begin{example}
На картинке \ref{pic:3-rooted-trees} изображен вид <<корневые
деревья с 3 вершинами>> (без какого-либо порядка на потомках).
\end{example}

\begin{figure}
\begin{center}
\begin{tikzpicture}
\draw (0pt, 0pt) -- (0pt, -20pt);
\draw (0pt, -20pt) -- (0pt, -40pt);
\draw[color=green, fill] (0pt,0pt) circle (6pt);
\draw[color=black, line width=1.5pt] (0pt,0pt) circle (7pt);
\draw[color=blue, fill] (0pt, -20pt) circle (6pt);
\draw[color=red, fill] (0pt, -40pt) circle (6pt);
\end{tikzpicture}
\begin{tikzpicture}
\draw (0pt, 0pt) -- (0pt, -20pt);
\draw (0pt, -20pt) -- (0pt, -40pt);
\draw[color=green, fill] (0pt,0pt) circle (6pt);
\draw[color=black, line width=1.5pt] (0pt,0pt) circle (7pt);
\draw[color=red, fill] (0pt, -20pt) circle (6pt);
\draw[color=blue, fill] (0pt, -40pt) circle (6pt);
\end{tikzpicture}
\begin{tikzpicture}
\draw (0pt, 0pt) -- (0pt, -20pt);
\draw (0pt, -20pt) -- (0pt, -40pt);
\draw[color=blue, fill] (0pt,0pt) circle (6pt);
\draw[color=black, line width=1.5pt] (0pt,0pt) circle (7pt);
\draw[color=green, fill] (0pt, -20pt) circle (6pt);
\draw[color=red, fill] (0pt, -40pt) circle (6pt);
\end{tikzpicture}
\begin{tikzpicture}
\draw (0pt, 0pt) -- (0pt, -20pt);
\draw (0pt, -20pt) -- (0pt, -40pt);
\draw[color=blue, fill] (0pt,0pt) circle (6pt);
\draw[color=black, line width=1.5pt] (0pt,0pt) circle (7pt);
\draw[color=red, fill] (0pt, -20pt) circle (6pt);
\draw[color=green, fill] (0pt, -40pt) circle (6pt);
\end{tikzpicture}
\begin{tikzpicture}
\draw (0pt, 0pt) -- (0pt, -20pt);
\draw (0pt, -20pt) -- (0pt, -40pt);
\draw[color=red, fill] (0pt,0pt) circle (6pt);
\draw[color=black, line width=1.5pt] (0pt,0pt) circle (7pt);
\draw[color=green, fill] (0pt, -20pt) circle (6pt);
\draw[color=blue, fill] (0pt, -40pt) circle (6pt);
\end{tikzpicture}
\begin{tikzpicture}
\draw (0pt, 0pt) -- (0pt, -20pt);
\draw (0pt, -20pt) -- (0pt, -40pt);
\draw[color=red, fill] (0pt,0pt) circle (6pt);
\draw[color=black, line width=1.5pt] (0pt,0pt) circle (7pt);
\draw[color=blue, fill] (0pt, -20pt) circle (6pt);
\draw[color=green, fill] (0pt, -40pt) circle (6pt);
\end{tikzpicture}
\begin{tikzpicture}
\draw (0pt, 0pt) -- (-20pt, -20pt);
\draw (0pt, 0pt) -- (20pt, -20pt);
\draw[color=red, fill] (0pt,0pt) circle (6pt);
\draw[color=black, line width=1.5pt] (0pt,0pt) circle (7pt);
\draw[color=blue, fill] (-20pt, -20pt) circle (6pt);
\draw[color=green, fill] (20pt, -20pt) circle (6pt);
\end{tikzpicture}
\begin{tikzpicture}
\draw (0pt, 0pt) -- (-20pt, -20pt);
\draw (0pt, 0pt) -- (20pt, -20pt);
\draw[color=blue, fill] (0pt,0pt) circle (6pt);
\draw[color=black, line width=1.5pt] (0pt,0pt) circle (7pt);
\draw[color=red, fill] (-20pt, -20pt) circle (6pt);
\draw[color=green, fill] (20pt, -20pt) circle (6pt);
\end{tikzpicture}
\begin{tikzpicture}
\draw (0pt, 0pt) -- (-20pt, -20pt);
\draw (0pt, 0pt) -- (20pt, -20pt);
\draw[color=green, fill] (0pt,0pt) circle (6pt);
\draw[color=black, line width=1.5pt] (0pt,0pt) circle (7pt);
\draw[color=red, fill] (-20pt, -20pt) circle (6pt);
\draw[color=blue, fill] (20pt, -20pt) circle (6pt);
\end{tikzpicture}
\end{center}
\caption{корневые деревья с 3 вершинами}
\label{pic:3-rooted-trees}
\end{figure}

Можно рассмотреть функтор $I:\Set \rightarrow Vect$, который сопоставляет
множеству векторное пространство, базис которого это множество.
Тогда $F \circ I: \B \rightarrow Vect$ --- сопоставляет каждому $n$
перестановочное представление группы $S_n$. При таком подходе, значение
характера этого представления $\chi(\sigma)$, это количество структур,
неподвижных относительно $\sigma \in S_n$.

\subsubsection{Сложение комбинаторных видов}
Сумму двух species $F$ и $G$ определим как поточечную сумму функторов.
На комбинаторном языке это будет означать <<либо структура типа $F$, либо
структура типа $G$>>. $(F + G)[n] = F[n] \coprod G[n]$ с покомпонентным
действием $S_n$.
\begin{example}
$\mathbb E = \mathbb E_e + \mathbb E_o$
\end{example}
\begin{example}
Любой вид $F$ можно разложить в такую сумму $F =
F_{1} + F_{2} + F_{3} + \dots$, где $F_{i}$ --- сужение $F$ на $i \in \mathcal
B$. Значение $F_{i}$ на $j \neq i$ равно $\emptyset$.
\end{example}

\subsubsection{Произведение комбинаторных видов} 
Определим произведение по Коши комбинаторных видов. По определению задать на
конечном множестве структуру типа $F \cdot G$ означает разбить множество точек 
на две части (всевозможные) и на первом ввести структуру типа $F$,
на втором --- типа $G$. 
$$(F \cdot G)[X] = \coprod\limits_{X_1 \coprod X_2 =
X}F[X_1] \times G[X_2]$$


C категорной точки зрения произведение по Коши возникает из тензорного
произведения на категории $\B$, которое на объектах задается как $n \otimes m =
(n + m)$, на морфизмах при помощи вложения $S_n \times S_m \hookrightarrow
S_{n+m}$ (все такие вложения сопряжены).
Известна конструкция свертки функторов из $\mathcal C$ в $\Set$, где
$\mathcal C$ --- моноидальная категория с копроизведениями
\href{http://nlab.mathforge.org/nlab/show/Day+convolution}{[Day
convolution \cite{Day}]}.


На языке теории представлений $F[n+m]$ как множество с действием группы
$S_{n+m}$ равно индуцированному представлению $Ind \uparrow_{S_n \times S_m}
^{S_{n+m}} F[n]\times F[m]$.

\begin{example}
$\E \times \E_1$ --- множество с выделенной точкой.
\end{example}
\begin{example}
$\mathbf C^2$ --- (упорядоченная) пара циклов.
\end{example}

\subsection{Композиция комбинаторных видов}
Кроме сложения и умножения на species можно ввести операцию композиции.
По определению задать на конечном множестве структуру типа $F \circ G$ означает
разбить множество точек на части (всевозможные), на частях (как новых точках)
ввести структуру типа $F$, а на каждой части --- типа $G$. Иначе говоря, <<раздуть>>
каждую точку структуры типа $F$ в структуру типа $G$.
$$
(F \circ G)[X] =
\coprod\limits_{\coprod\limits_i X_i = X} F[\{X_i\}_i] \times
(\coprod\limits_{i} G[X_i]) 
$$

\begin{remark}
\label{rem:finit}
Определение species не предполагает конечности $F[n]$, однако цикленный индекс
(см. раздел \ref{sec:cycle}) можно писать только для таких видов. Класс таких
species не замкнут относительно композиции. Поскольку при подстановке
species $F$, для котогоро $F[0] \neq \emptyset$ можно выделить сколько угодно пустых
частей. Поэтому в дальнейшем в записи $F \circ G$, мы будем неявно предполагать
что внутренний операнд <<сужен>> на $\mathbb N_{+}$.
\end{remark}

\begin{example}
$\E_1 \circ F = F$, $F \circ \E_1 = F$. $\E_1$ является нейтральным элементом в
моноиде species по композиции.
\end{example}
\begin{example}
$\E_2 \circ \mathbf C$ --- (неупорядоченная) пара циклов.
\end{example}
\begin{example}
$\E \circ \E$ --- структура разбиения множества.
\end{example}
\begin{example}
$\E \circ \mathbf C = \mathbf S$ --- структура перестановки.
Буквально перестановка --- это набор циклов.
\end{example}

Для того, чтобы ввести композицию на категорном языке нам
понадобится дополнительная конструкция: аналитический функтор.
\subsubsection{Аналитический функтор комбинаторных видов}
Аналитический функтор (введен Жуаялем в \cite{J2}) $\mathcal F$ соответствует
species $F$. Вводить его можно разными способами, мы ограничимся универсальным
свойством и явной конструкцией. 
\begin{definition}
Аналитический функтор является левым расширением по Кану функтора $F$
относительно $i$.

\begin{tikzpicture}
\label{comm:an}
	\node (B) {$\B$};
	\node (S1) [below of=B] {$\Set$};
	\node (S2) [right of=B, node distance=3cm] {$\Set$};
	\draw [right hook->] (B) to node [swap] {$i$} (S1);
	\draw [->] (B) to node {$F$} (S2);
	\draw [->] (S1) to node [swap] {$\mathcal F$} (S2);	
\end{tikzpicture}
\end{definition}

Эта диаграмма не является коммутативной, а коммутативна лишь настолько,
насколько может быть коммутативной диаграмма подобного вида. А именно,
существует естественное преобразование  $\kappa \colon F \rightarrow i \circ
\mathcal F$, обладающее следующим универсальным свойством:
для любого функтора $M \colon \Set \rightarrow \Set$ и морфизма функторов $\eta
\colon F \rightarrow i \circ M$ этот морфизм пропускаеться через $\mathcal F$ при помощи $\kappa$.

\begin{tikzpicture}
\label{comm:an-uni}
	\node (F) {$F$};
	\node (Fm) [right of=F, node distance=3cm] {$\mathcal F$};
	\node (M) [below of=Fm] {$M$};
	\draw [->] (F) to node {$\kappa$} (Fm);
	\draw [->] (F) to node [swap] {$\alpha$} (M);
	\draw [->, dashed] (Fm) to node [swap] {} (M);
\end{tikzpicture}


Явная конструкция для аналитического функтора. Доказательство см. в
\cite{BergTrees}.
\begin{equation}
\label{eq:an}
	\mathcal F(A) = \sum\limits_n F[n] \times A^n / S_n
\end{equation}

\begin{remark}
\label{rem:color}
У аналитического функтора для типа структуры $F$ имеется прозрачная
комбинаторная интерпретация.
Если трактовать множество $A$ как набор цветов,
то значение аналитического функтора $\mathcal F(A)$ трактуется как множество структур типа $F$
раскрашенных в цвета из $A$.
\end{remark}

\subsubsection{Композиция аналитических функторов комбинаторных видов}
\begin{theorem}
Композиция аналитических функторов $\mathcal F \circ \mathcal G$ является
аналитическим функтором для $F \circ G$.
\end{theorem}
\begin{proof}
Набросок. Согласно конструкции $\mathcal F(\mathcal
G (A)) = \sum\limits_k F[k] \times (\sum\limits_m G[m] \times A^m / S_m)^k /
S_k = \sum\limits_n \sum\limits_{k, m_1 + \dots + m_k = n} F[k] \times
(\coprod\limits_{i} G[m_i]) \times A^n / S_n$.

Строгое доказательство см. в \url{http://arxiv.org/pdf/math/9811127v1.pdf}
(Lemma 2.5) [это еще и про цикленные индексы]
\end{proof}

\subsubsection{Другой взгляд на композицию аналитических функторов}
Дело в том, что композиция комбинаторных видов --- это в некотором смысле
частный случай аналитического функтора, только не со значениями в $Set$,
а со значениями в $\Species = \Hat \B$.

А именно, для любой моноидальной категории $\mathcal C$ со всевозможными
копределами и объекта $A$ из $\mathcal C$, функтор из $\mathcal B$ в $\mathcal
C$, который $1$ отправляет в $A$, а $n$ отправляет в $A^{\otimes n}$ (тензорная
степень), можно <<продолжить по непрерывнoсти>> (видимо расширение Кана) до
функтора из категории предпучков на $\mathcal B$ со значениями в $\mathcal C$
\cite{Durov}. В этой работе Дурова этот фнктор обозначается $\Phi_A$.

В случае когда $\mathcal C = Set$, получаем
$\Phi_A(F)=\mathcal F(A)$. В случае, когда $\mathcal C = \Species$, получаем
$\Phi_G(F)=F \circ G$. То есть можно определить композицию, как функтор
$G \mapsto F\circ G$ при помощи $\Phi_G$. Надо только проверить функториальность
по $G$ (в обозначении $\Phi_G$, буква $G$ обозначает постоянный параметр, а в
аналитеческом функторе --- это переменный аргумент).
При таком взгляде на подстановку теорема о композиции становится почти тавтологией.

\subsubsection{Цикленный индекс}
\label{sec:cycle}
%Процедура декатегорификации не имеет строго математического смысла, так же как
%и процедура квантования.
%Сейчас мы предложим процедуру, которая, стартуя с обычных species,
%на выходе дает классический цикленный индекс/фробениусову характеристику.
%Затем мы попытаемся аналогические действия провести и в гипероктаэдральном
% случае.
%Декатегорификацией моноидальной категории $\mathbb B$ является моноид классов
% изоморфизма объектов категории $\mathbb \B$, то есть моноид натуральных чисел по сложению.
%Декатегорификкацией $\widehat{\mathbb \B}$ естественным образом оказывается
%моноидная алгебра с коэффициентами из $\mathbb Z$ для моноида $\mathbb N$, то
% есть кольцо многочленов $Z[X]$.
%(Правда это не то, что мы хотели. Чтобы получить цикленный индекс надо
% декатегорифицировать саму операцию подстановки и аналитический функтор).
% [TODO: этот таинственный абзац стоит переписать]

Будем рассматривать только species, конечные в смысле замечания
\ref{rem:finit}.
Хочется используя идею подсчета с весом, написать аналог производящей функции
для покрашенных (в смысле замечания \ref{rem:color}) структур. Цветам
сопоставим переменные $x_1, x_2, x_3, \dots$. Раскрашенной структуре с
раскраской $\{x_{i_1}, x_{i_2}, \dots, x_{i_k}\}$ сопоставим моном $x_{i_1} x_{i_2} \dots x_{i_k}$.  
Например, расскраске в которой 2 первых цвета и 1 второй соответсвует моном
$x_1^2x_2$. 
Суммируя по всем расскрашенным структурам из формулы \ref{eq:an}, мы получим
симметрическую (от $x_1, x_2, x_3, \dots$) функцию. Коэффициент
при каждом мономе --- это число раскрашенных структур с заданной расскраской. 

Введем некоторые обозначения 
$\lambda = (\lambda_1, \lambda_2, \lambda_3, \dots) \vdash n$ --- разбиение.
$\sigma$ --- перестановка цикленного типа $\lambda$. $z_\lambda$ --- индекс
класса сопряженности $\sigma$.
$\chi$ --- характер (перестановочного) представления заданного $F$.
$\psi^{\lambda} = 
(x_1^{\lambda_1} + x_2^{\lambda_1} + x_3^{\lambda_1} + \dots)
(x_1^{\lambda_2} + x_2^{\lambda_2} + x_3^{\lambda_2} + \dots)
(x_1^{\lambda_3} + x_2^{\lambda_3} + x_3^{\lambda_3} + \dots)
\dots$.

\begin{statement}
Фробениусовой характеристикой или цикленным индексом species $F$ будем называть
симметрическую функцию:
\begin{equation}
\label{eq:fr}
\mathcal Z_F =
\sum_{n}\frac{1}{n!}\sum_{\sigma \in S_n}\chi(\sigma)\psi^{\lambda(\sigma)} =
\sum_{n, \lambda \vdash n}\chi(\sigma_{\lambda})
\frac{\psi^{\lambda}}{z_{\lambda}}
\end{equation}
Коэффициент при мономе $x_{i_1} x_{i_2} \dots x_{i_k}$ равен числу
раскрашенных структур с расскраской $\{x_{i_1}, x_{i_2},
\dots, x_{i_k}\}$.

\end{statement}
\begin{proof}
По Лемме Бернсайда количество орбит равно усредненному по всем элементам
группы числу неподвижных точек. Чтобы раскрашенная структура была неподвижна под
действием перестановки $\sigma$ нужно, чтобы во-первых она была неподвижна как
нераскрашенная структура, а во-вторых расскраска должна переходить в себя.
Тогда первое условие дает нам сомножитель
$\chi(\sigma)$. Второе условие требует покраски каждого
цикла в один и тот же цвет.
\end{proof}

\begin{example}
$\mathcal Z_{\E_1} = \psi^1$
\end{example}

\begin{example}
$\mathcal Z_{\E_2} = \frac{1}{2}((\psi^1)^2 + \psi^2)$
\end{example}

\begin{example}
$\mathcal Z_{\E} = e^{(\psi^1 + \frac{\psi^2}{2} + \frac{\psi^3}{3} +
\dots)}$. Доказательство смотри в \cite{BergTrees}.
\end{example}

\subsubsection{Плетизм цикленных индексов}
\begin{theorem}
Композиции аналитических функторов соответствует плетизм цикленных
индексов.
\end{theorem}
Чудесный факт заключается в том, что в декатегорификации
композиция соответствует простой формуле подстановки. Сейчас мы ее напишем и
приведем набросок доказательства. В качестве множества цветов $A$ рассмотрим
счетный набор цветов $x_1, x_2, x_3, \dots$ Цикленный индекс запишем
относительно базиса кольца симметрических функций $\psi^1, \psi^2, \psi^3, \dots$
\begin{statement}
\label{th:compos}
\begin{multline}
\label{eq:zfg}
	\mathcal Z_{F \circ G} (\psi^1, \psi^2, \psi^3, \dots) = \\
	\mathcal Z_F(
		\mathcal Z_G(\psi^1, \psi^2, \psi^3, \dots),
		\mathcal Z_G(\psi^2, \psi^4, \psi^6, \dots),
		\mathcal Z_G(\psi^3, \psi^6, \psi^9, \dots),
		\dots
	)
\end{multline}
\end{statement}
\begin{proof}
В композиции двух аналитических функторов получается, что цвета в которые мы
красим структуру $F$ это структуры типа $G$. То есть $\mathcal Z_{F \circ G} =
\mathcal Z_F(\psi_g^1, \psi_g^2, \psi_g^3, \dots)$, где $\psi_g^i = (g_1^i +
g_2^i + g_3^i + \dots)$, где $g_i$ --- перечисление всех структур типа $G$.
Нужно раскрыть переменные $g_i $ --- написать их относительно начальных цветов.
Формулу $\psi_g^i = \mathcal Z_G(\psi^i, \psi^{2i}, \psi^{3i}, \dots)$ легко
понять в переменных $x_1, x_2, x_3, \dots$. Мы должны покрасить $i$ кусков в
одну и ту же $G$--структуру. Значит каждый цвет $x_j$ в $\mathcal Z_G$ при
подстановке в $\psi^i$ заменяется на $x_j^i$.
\end{proof}
\begin{remark}
Формулу \ref{eq:zfg} можно специализировать для подсчета labeled--структур. То
есть покрашенных структур у которых нет двух одинаковых цветов в расскраске.
Соответсвующие мономы (в базисе $x_1, x_2, x_3, \dots$) возникают только при
раскрытии мономов вида $c(\psi^1)^k$ и коэффициент в них равен $ck!$ --- такой
же как при мономе с точностью до факториала. Этот факториал приводит к
необходимости рассматривать экспоненциальные производящие функции вместо
обычных. Можно занулить все остальные мономы подстановкой $\psi^1 = t, \psi^2 =
0, \psi^3 = 0, \psi^4 = 0$. Формула \ref{eq:zfg} примет вид $
\mathcal Z_{F \circ G} (t, 0, 0, \dots) =
	\mathcal Z_F(
		\mathcal Z_G(t, 0, 0, \dots), 0, 0, \dots
	)
$.
А значит для экспоненциальных производящих функции labeled-структур справедливо
равенство
\begin{equation}
\label{eq:comp}
(f \circ g) (t) = f(g(t))
\end{equation}
\end{remark}

\begin{example}
(Экспоненциальная) производящая функция для $\E$ это $e^x = 1 + x +
\frac{1}{2!}x^2 + \frac{1}{3!}x^3 + \dots$. А производящая функция для
непустых циклов $\mathbf C$ это $-log(1-x) = x + \frac{1}{2}x^2 +
\frac{1}{3}x^3 + \dots$.
А для $\mathbf S$ производящая функция это $\frac{1}{1-x} = 1 + x + x^2 + x^3 +
\dots$.
И действительно $e^{-log(1-x)} = \frac{1}{1-x}$.
\end{example}

        \section{Гипероктаэдральные комбинаторные виды}
\subsection{Определение}
Рассмотрим категорию $\HSet$. В ней объекты это множества, снабженные
дополнительным действием --- инволюцией. А стрелки, это морфизмы, сохраняющие инволюцию. 
Рассмотрим категорию $\HB$ --- подкатегорию конечных множеств из
$\HSet$ с морфизмами только биекциями, и инволюциями без неподвижных точек.
Функтор $F:\HB \rightarrow \HSet$ --- гипероктаэдральный (или кубический)
это комбинаторный вид. Мы будем так же для краткости употреблять термин
\emph{h-species}. Группоид $\HB$ эквивалентен группойду, объекты которого $\Bar
n = \{-n, -n+1, \dots, -1, 1, 2, \dots, n-1, n\}$, инволюция - смена знака. 
Эпитет гипероктаэдральный используется потому, что на $\HB[\Bar n]$  действует
гипероктаэдральная группа $B_n$ --- группа движений n-мерного куба (иногда будем обозначать ее $H_n$).
Некоторая не очень ясная комбинаторная интерпретация: множеству граней куба
сопоставляется множество структур на этих гранях, а действие $B_n$ возникает из
перестановок граней.

\begin{example}
Вид $\mathbb H$ --- структура куб. Он сопоставляет $\Bar n$ одно множество. Все
элементы $B_n$ переходят в тождественное отображение.
\end{example}
\begin{example}
$\dA$ --- неразличимая пара граней ($\mathbb H_1$). $\dB$ --- различимая пара
граней.
Оба они принимают значение $\emptyset$ на всем, кроме $\Bar 1$. Второе
соответствует действию $H_1$ на 2-х точечном множестве.
\end{example}
\begin{example}
Аналогично $\dAA$ --- структура куб размерности 2 ($\mathbb H_2$). $\dBB$ ---
куб размерности 2 с различимыми противоположными гранями.  Второе
соответствует действию $H_2$ на 4-х точечном множестве.
\end{example}
\begin{example}
Структура $\dB \times \dB$. Это не то же самое что \dBB, поскольку это <<упорядоченная пара \dB>>.
\end{example}

\subsection{Вложение species в h-species}
Обычные комбинаторные виды можно <<вложить>> в гипероктаэдральные. Иными
словами можно каждый species рассмотреть как h-species. Для этого достаточно
рассматривать структуру не на точках, а на парах (неразличимых) граней. Если
$F:\B \rightarrow \Set$, то $\tilde{F}: \HB \rightarrow \HSet$, где
$\tilde{F}[\Bar n] = F[n]$ как множество, а инволюция тождественна.

\subsection{Сложение и умножение h-species}
Сложение и умножение определяются полностью аналогично species и тут проблем не
возникает.


\subsection{Аналитический функтор для h-species}
Хочется построить аналог аналитического функтора для h-species

\begin{tikzpicture}
\label{comm:h-an}
	\node (B) {$HB$};
	\node (S1) [below of=B] {$HSet$};
	\node (S2) [right of=B, node distance=3cm] {$HSet$};
	\draw [right hook->] (B) to node [swap] {$i$} (S1);
	\draw [->] (B) to node {$F$} (S2);
	\draw [->] (S1) to node [swap] {$\mathcal F$} (S2);
\end{tikzpicture}

\begin{equation}
\label{eq:h-an}
	\mathcal F = \sum\limits_n F[\Bar n] \times A^{\Bar n} / B_n
\end{equation}
Где $A^{\Bar n}$ задает отображение, сохраняющее инволюцию. 
TODO:Здесь нужно добавить проверку универсальности картинки

\begin{definition}
Будем называть элементы $(F[\Bar n] \times
A^{\Bar n})$ $A$-крашенными $F$ структурами на гранях $n$-мерного куба. Таким
образом правую часть \ref{eq:h-an} можно интерпретировать как всевозможные
классы эквивалентности крашеных структур.
\end{definition}

\subsection{Декатегорификация аналитического функтора} 
Можно действовать наивно: написать производящую функцию для числа раскрасок, по
аналогии с классическим случаем. Такая формула (\ref{eq:h-fr1}) рассматривалась
(в контексте теории представлений группы $S_n \wr G$) в работе
\url{http://www.combinatorics.org/ojs/index.php/eljc/article/download/v11i1r56/pdf}
(см. также приложение B во втором анлгийском издании книги Макдональда
\cite{Mac2}). Но при таком подходе наши попытки определить гипероктаэдральных
плетизм оказались безуспешны. Выяснилось, что правильный аналог цикленного
индекса должен помнить информацию о следующем свойстве раскрашенной структуры.

Отметим, что раскраска (элемент $A^{\Bar n}$), это отображение,
сохраняющее инволюцию. Значит пара элементов $(-i, i)$
отображается либо в один и тот же элемент $(a, a)$ (который инволюцией
переводиться в себя), либо в пару элементов $(b, \Bar b)$, сопряженных
инволюцией. Будем называть первый случай \emph{моноцветом}, второй ---
\emph{бицветом}.

Покрашенные структуры сами по себе можно рассматривать как моноцвет, либо
бицвет. Это по--прежнему определяется длинной орбиты инволюции на $A$, уже
после факторизации по $B_n$. То есть кроме действия $B_n$ есть еще внешняя
инволюция --- действие $Z_2$. Будем разделять расскрашенные структуры на
\emph{моноструктуры} и \emph{биструктуры}.

Цикленный индекс, считающий только моноструктуры будем обозначать
$\mathcal Z^{(1)}$, биструктуры --- $\mathcal Z^{(2)}$. 
\begin{remark}
Таким образом, цикленный индекс для h-species представляет собой пару $(\Zone,
\Ztwo)$.
\end{remark}
\begin{statement}
Количество орбит под действием $H_n \times Z_2$ соответствует $\mathcal Z^{(1)}
+ \mathcal Z^{(2)}$, а под действием только $H_n$ соответствует $\mathcal Z^{(1)} + 2\mathcal
Z^{(2)}$. 
\end{statement}
\begin{proof}
В первом случае каждая моноструктура и биструктура будет посчитана 1 раз.
А во втором каждая биструктура будет посчитан два раза, т.к. действие инволюции
<<склеивающей>> две части биструктуры не учтено.
\end{proof}

\subsubsection{Подсчет цикленного индекса}
В качестве H-множества цветов возьмем счетное множество моноцветов $x_1, x_2,
x_3, \dots$ объединенное с счетным множеством бицветов $y_1, y_2, y_3, \dots$.

Допустим, что мы придумали весовую функцию, отправляющую каждую расскрашенную
структуру в моном и любая орбита отправляеться в один моном. Применив Лемму
Бернсайда переходим к подсчету неподвижных точек. Циклы в каждом элементе $H_n$
бывают двух типов:
\emph{длинные} --- каждая грань входит в цикл вместе со своей противоположной
гранью и \emph{короткие} --- пара граней лежит в симметричных, различных циклах. 

Посчитаем количество неподвижных точек для $H_n$. Пусть $\lambda^1$~
--- цикленный тип коротких перестановок, $\lambda^2$ --- цикленный тип длинных
перестановок. 
\begin{statement}
Неподвижные раскрашенные структуры, это в точности
те, у которых длинный цикл покрашен в моноцвету, а пара симметричных коротких 
может быть покрашена либо в моноцвет, либо в бицвет. 
\end{statement}
Под покрашенным циклом мы подразумеваем покраску всех его элементов в этот цвет
(такая покрашенная структура будет неподвижна относительно действия этого
элемента $H_n$).

\begin{statement}
Справедлива формула:
\begin{equation}
\label{eq:h-fr1}
\begin{split}
\mathcal Z_F^{(1)} + 2\mathcal Z_F^{(2)} = 
\sum_{n}\frac{1}{2^{n}n!}\sum_{\sigma \in B_n}&\chi(\sigma)
\psi_{x, y, y}^{\lambda^1(\sigma)} \psi_{x}^{\lambda^2(\sigma)} = \\
\sum_{n, \lambda^1 + \lambda^2 \vdash n}&\chi(\sigma_{\lambda^1 \lambda^2})
\frac{\psi_{x, y, y}^{\lambda^1} \psi_{x}^{\lambda^2}}{z_{\lambda^1 \lambda^2}}
\end{split}
\end{equation}
\end{statement}
Здесь нижний индекс $\psi$ означает переменные по которым берется степенная
сумма. Например $\psi_{x, y, y}^2 =  (x_1^2 + x_2^2 + x_3^2 + \dots + y_1^2 +
y_2^2 + y_3^2 + \dots + y_1^2 + y_2^2 + y_3^2 + \dots)$. При этом коофициент
2 у $y_i^2$ отражает тот факт, что можно раскрасить $k$ пар граней в бицвет,
так чтобы расскраска была неподвижна, под действием короткого цикла, 2-мя способами.

Посчитаем количество неподвижных точек для $H_n \times Z_2$. Разобъем
сумму на две части --- $(h, \Bar 0)$ и $(h, \Bar 1)$. Для первой формула будет
аналогична \ref{eq:h-fr1}, только из-за того что порядок группы в 2 раза больше,
появится коофициент $\frac{1}{2}$. 
Во второй части по-прежнему можно красить и длинные и короткие циклы в моноцвет.
А вот с бицветом происходит любопытная вещь --- предположим мы красим в него
цикл (пару циклов, в случае короткого). Тогда реальный цикл от $(h, \Bar 1)$
будет получатся из циклов $h$ добавлением <<смены грани>> на каждом шаге. 
Значит для циклов нечетной длинны сменится свойство короткий--длинный. 
\begin{example}
Пусть $h_e = 
\begin{tikzpicture}
\draw [<->] (5pt, 0pt) -- (15pt, 0pt);
\draw [<->] (5pt, 20pt) -- (15pt, 20pt);
\draw[fill] (0pt,0pt) circle (3pt);
\draw[fill] (0pt, 20pt) circle (3pt);
\draw[fill] (20pt, 0pt) circle (3pt);
\draw[fill] (20pt, 20pt) circle (3pt);
\end{tikzpicture}
$.
Тогда $(h_e, \Bar 1) = 
\begin{tikzpicture}
\draw [<->] (4pt, 4pt) -- (16pt, 16pt);
\draw [<->] (4pt, 16pt) -- (16pt, 4pt);
\draw[fill] (0pt,0pt) circle (3pt);
\draw[fill] (0pt, 20pt) circle (3pt);
\draw[fill] (20pt, 0pt) circle (3pt);
\draw[fill] (20pt, 20pt) circle (3pt);
\end{tikzpicture}
$
\end{example}
\begin{example}
Пусть $h_o = 
\begin{tikzpicture}
\draw [->] (5pt, 0pt) -- (15pt, 0pt);
\draw [->] (5pt, 20pt) -- (15pt, 20pt);
\draw [->] (25pt, 0pt) -- (35pt, 0pt);
\draw [->] (25pt, 20pt) -- (35pt, 20pt);
\draw [->] (40pt, -5pt) to [out=-135,in=-45] (0pt, -5pt);
\draw [->] (40pt, 25pt) to [out=135,in=45] (0pt, 25pt);
 
\draw[fill] (0pt,0pt) circle (3pt);
\draw[fill] (0pt, 20pt) circle (3pt);
\draw[fill] (20pt, 0pt) circle (3pt);
\draw[fill] (20pt, 20pt) circle (3pt);
\draw[fill] (40pt, 0pt) circle (3pt);
\draw[fill] (40pt, 20pt) circle (3pt);
\end{tikzpicture}
$.
Тогда $(h_o, \Bar 1) = 
\begin{tikzpicture}
\draw [->] (4pt, 4pt) -- (16pt, 16pt);
\draw [->] (4pt, 16pt) -- (16pt, 4pt);
\draw [->] (24pt, 4pt) -- (36pt, 16pt);
\draw [->] (24pt, 16pt) -- (36pt, 4pt);
\draw [->] (45pt, 20pt) to [out=0,in=180] (-5pt, 0pt);
\draw [->] (45pt, 0pt) to [out=0,in=180] (-5pt, 20pt);
 
\draw[fill] (0pt,0pt) circle (3pt);
\draw[fill] (0pt, 20pt) circle (3pt);
\draw[fill] (20pt, 0pt) circle (3pt);
\draw[fill] (20pt, 20pt) circle (3pt);
\draw[fill] (40pt, 0pt) circle (3pt);
\draw[fill] (40pt, 20pt) circle (3pt);
\end{tikzpicture}
$
\end{example}

\begin{statement}
Справедлива формула:
\begin{equation}
\label{eq:h-fr2}
\begin{split}
\mathcal Z_F^{(1)} + \mathcal Z_F^{(2)} = 
\frac{1}{2}&
\sum_{n, \lambda^1 + \lambda^2 \vdash n}\chi(\sigma_{\lambda^1 \lambda^2})
\frac{\psi_{x, y, y}^{\lambda^1} \psi_{x}^{\lambda^2}}{z_{\lambda^1 \lambda^2}}
+ \\
\frac{1}{2}&
\sum_{n, \lambda_o^1 + \lambda_o^2 + \lambda_e^1 + \lambda_e^2 \vdash
n}\chi(\sigma_{\lambda_o^1 \lambda_o^2 \lambda_e^1 \lambda_e^2})
\frac{\psi_{x, y, y}^{\lambda_e^1 + \lambda_o^2} \psi_{x}^{\lambda_e^2 + 
\lambda_o^1}}{z_{\lambda_o^1 \lambda_o^2 \lambda_e^1 \lambda_e^2}}
\end{split}
\end{equation}
\end{statement}
Где $\lambda_o$ --- циклы нечетной длинны, $\lambda_e$ ---
циклы четной длинны.

Из формул \ref{eq:h-fr1}, \ref{eq:h-fr2} легко получить
\begin{equation}
\mathcal Z_F^{(1)} = 
\sum_{n, \lambda_o^1 + \lambda_o^2 + \lambda_e^1 + \lambda_e^2 \vdash
n}\chi(\sigma_{\lambda_o^1 \lambda_o^2 \lambda_e^1 \lambda_e^2})
\frac{\psi_{x, y, y}^{\lambda_e^1 + \lambda_o^2} \psi_{x}^{\lambda_e^2 + 
\lambda_o^1}}{z_{\lambda_o^1 \lambda_o^2 \lambda_e^1 \lambda_e^2}}
\end{equation}

\begin{equation}
\begin{split}
\mathcal Z_F^{(2)} = 
\frac{1}{2}&
\sum_{n, \lambda^1 + \lambda^2 \vdash n}\chi(\sigma_{\lambda^1 \lambda^2})
\frac{\psi_{x, y, y}^{\lambda^1} \psi_{x}^{\lambda^2}}{z_{\lambda^1 \lambda^2}}
- \\
\frac{1}{2}&
\sum_{n, \lambda_o^1 + \lambda_o^2 + \lambda_e^1 + \lambda_e^2 \vdash
n}\chi(\sigma_{\lambda_o^1 \lambda_o^2 \lambda_e^1 \lambda_e^2})
\frac{\psi_{x, y, y}^{\lambda_e^1 + \lambda_o^2} \psi_{x}^{\lambda_e^2 + 
\lambda_o^1}}{z_{\lambda_o^1 \lambda_o^2 \lambda_e^1 \lambda_e^2}}
\end{split}
\end{equation}

\subsubsection{Примеры вычисления цикленного индекса}
Посчитаем цикленные индексы для простых h-species.
Здесь мы будем писать $Z(A)$ вместо $Z_A$. Это не должно вызывать путаницу,
поскольку вместо $A$ будут использоваться схематические картинки. Их
никак не перепутать с переменными, от которых считается цикленный индекс. 
\begin{example}
$$
\Zfull(\dA) = \frac{1}{2}(\psi_{x,y,y}^1 + \psi_{x}^1) = \psi_{x,y}^1
$$
$$
\mathcal Z^{(1)}(\dA) = \frac{1}{2}(\psi_{x}^1 + \psi_{x, y, y}^1) = \psi_{x,y}^1
$$
Значит
$$
\mathcal Z^{(2)}(\dA) = 0
$$
\end{example}
\begin{example}
$$
\Zfull(\dB) = \frac{1}{2}(2\psi_{x,y,y}^1 + 0\psi_{x}^1) = \psi_{x,y,y}^1
$$
$$
\mathcal Z^{(1)}(\dB) = \frac{1}{2}(2\psi_{x}^1 + 0\psi_{x, y, y}^1) = \psi_{x}^1
$$
Значит
$$
\mathcal Z^{(2)}(\dB) = \psi_{y}^1
$$
\end{example}
\begin{example}
$$
\Zfull(\dAA) = \frac{1}{8}((\psi_{x,y,y}^1)^2 + (\psi_{x}^1)^2 + 2\psi_{x}^2 +
2(\psi_x^1\psi_{x,y,y}^1) + 2\psi_{x,y,y}^2)
$$
Здесь коофициенты --- не характеры (характер при каждом слагаемом $= 1$).
$$
\mathcal Z^{(1)}(\dAA) = \frac{1}{8}((\psi_{x}^1)^2 + (\psi_{x, y, y}^1)^2 +
2\psi_{x, y, y}^2 + 2(\psi_{x, y, y}^1\psi_{x}^1) + 2\psi_{x}^2) = \Zfull(\dAA)
$$
Последнее следовало и из общих соображений: легко видеть что $\mathcal
Z^{(2)}(\dAA) = 0$.
\end{example}
\begin{example}
$$
\Zfull(\dBB) = \frac{1}{8}(4(\psi_{x,y,y}^1)^2 + 0(\psi_{x}^1)^2 + 0\psi_{x}^2
+ 0(\psi_x^1\psi_{x,y,y}^1) + 2\times2\psi_{x,y,y}^2)
$$
$$
\mathcal Z^{(1)}(\dBB) = \frac{1}{8}(4(\psi_{x}^1)^2 + 2\times2\psi_{x,y,y}^2)
$$
Откуда
$$
\mathcal Z^{(2)}(\dBB) = \frac{1}{2}(\Zfull(\dBB) - \mathcal
Z^{(1)}(\dBB)) = \frac{1}{2}(\psi_{y,y}^1\psi_{x}^1 +
\frac{1}{2}(\psi_{y, y}^1)^2) = \psi_{y}^1\psi_{x}^1 + (\psi_{y}^1)^2 $$
\end{example}

\subsection{Сумма и произведение цикленных индексов}
\subsubsection{Сумма}
Сумма цикленных индексов соответсвует поточечной сумме аналитических
функторов и здесь нет никаких сюрпризов:
$$
\mathcal Z_{A + B}^{(1)} = \mathcal Z_A^{(1)} + \mathcal Z_B^{(1)}
$$
$$
\mathcal Z_{A + B}^{(2)} = \mathcal Z_A^{(2)} + \mathcal Z_B^{(2)}
$$
\subsubsection{Произведение}
Для произведения уже не совсем так. 
\begin{statement}
Моноструктура получается
в произведении двух моноструктур. А биструктура получается, если один из
сомножителей биструктура. Причем в случае, когда оба сомножителя ---
биструктуры, получается две различных биструктуры. 
\end{statement}
То есть
$$
\mathcal Z_{A * B}^{(1)} = \mathcal Z_A^{(1)} * \mathcal Z_B^{(1)}
$$
$$
\mathcal Z_{A * B}^{(2)} = 
\mathcal Z_A^{(1)} * \mathcal Z_B^{(2)} + 
\mathcal Z_A^{(2)} * \mathcal Z_B^{(1)} +
2 (\mathcal Z_A^{(2)} * \mathcal Z_B^{(2)})
$$
Откуда следует
$$
(\mathcal Z_{A * B}^{(1)} + 2\mathcal Z_{A * B}^{(2)}) = 
(\mathcal Z_A^{(1)} + 2\mathcal Z_A^{(2)}) * 
(\mathcal Z_B^{(1)} + 2\mathcal Z_B^{(2)})
$$

\begin{remark}
Это логично, поскольку $(\mathcal Z_F^{(1)} + 2\mathcal Z_F^{(2)})$ --- это
цикленный индекс для цветов, с <<забытой>> инволюцией.
\end{remark}

\subsubsection{Примеры цикленных индексов произведений}
Посчитаем произведение уже известных h-структур и их цикленных индексов.

\begin{example}
Структура $\dA \times \dA$.
$$
\mathcal Z^{(1)}(\dA \times \dA) = \mathcal Z^{(1)}(\dA) \times \mathcal
Z^{(1)}(\dA) = (\psi_{x, y}^1)^2
$$
\end{example}
\begin{example}
Структура $\dB \times \dB$.
$$
\Zfull(\dB \times \dB) = \Zfull(\dB) \times \Zfull(\dB) =(\psi_{x, y, y}^1)^2
$$
Легко получить эту же формулу и прямым подсчетом по формуле \ref{eq:h-fr1}, как
$\frac{1}{8}(8(\psi_{x, y, y}^1)^2)$.
$$
\mathcal Z^{(1)}(\dB \times \dB) = \mathcal Z^{(1)}(\dB) \times \mathcal
Z^{(1)}(\dB) =(\psi_{x}^1)^2
$$
\end{example}

\subsection{Цикленный индекс композиции}
Теперь попробуем выстроить теорию композиции цикленного индекса для h-species,
параллельно теории species. В качестве моноцветов возьмем $\{x_1, x_2, x_3,
\dots\}$, бицветов --- $\{y_1, y_2, y_3, \dots\}$. Формулы \ref{eq:h-fr1} и
\ref{eq:h-fr2} подсказывают, что в качестве симметричного базиса можно брать не
$\{\psi_x^i, \psi_y^j\}$ а $\{\psi_x^i, \psi_{x,y,y}^j\}$. Впрочем это
тривиальная замена переменных.

\begin{problem}
Как выразить
\begin{equation*}
\begin{split}
\mathcal Z^{(1)}_{F \circ G} (&\psi_x^1, \psi_x^2, \psi_x^3, \dots, \\
						&\psi_{x,y,y}^1, \psi_{x,y,y}^2, \psi_{x,y,y}^3, \dots)
\end{split}
\end{equation*}

\begin{equation*}
\begin{split}
\mathcal Z^{(2)}_{F \circ G} (&\psi_x^1, \psi_x^2, \psi_x^3, \dots, \\
						&\psi_{x,y,y}^1, \psi_{x,y,y}^2, \psi_{x,y,y}^3, \dots)
\end{split}
\end{equation*}
\end{problem}

\begin{theorem}
\begin{equation}
\begin{split}
\label{eq:h-zfg}
	\mathcal Z^{(i)}_{F \circ G} (\psi_x^1, \psi_x^2, \psi_x^3, &\dots, 
	\psi_{x,y,y}^1, \psi_{x,y,y}^2, \psi_{x,y,y}^3, \dots) = \\
	\mathcal Z_F^{(i)}(
		&\mathcal Z^{(1)}_G(\psi_x^1, \psi_x^2, \psi_x^3, \dots, 
					 \psi_{x, y, y}^1, \psi_{x, y, y}^2, \psi_{x, y, y}^3, \dots), \\
		&\mathcal Z^{(1)}_G(\psi_x^2, \psi_x^4, \psi_x^6, \dots, 
					 \psi_{x, y, y}^2, \psi_{x, y, y}^4, \psi_{x, y, y}^6, \dots), \\
		&\mathcal Z^{(1)}_G(\psi_x^3, \psi_x^6, \psi_x^9, \dots, 
					 \psi_{x, y, y}^3, \psi_{x, y, y}^6, \psi_{x, y, y}^9, \dots), \\
		&\dots, \\
		&\Zfull_G(\psi_x^1, \psi_x^2, \psi_x^3, \dots, 
					 \psi_{x,y,y}^1, \psi_{x,y,y}^2, \psi_{x,y,y}^3, \dots), \\
		&\Zfull_G(\psi_x^2, \psi_x^4, \psi_x^6, \dots, 
					 \psi_{x,y,y}^2, \psi_{x,y,y}^4, \psi_{x,y,y}^6, \dots), \\
		&\Zfull_G(\psi_x^3, \psi_x^6, \psi_x^9, \dots, 
					 \psi_{x,y,y}^3, \psi_{x,y,y}^6, \psi_{x,y,y}^9, \dots), \\
		&\dots
	)
\end{split}	
\end{equation}
Эта формула слишком грамоздкая, поэтому напишем ее на уровне членов:
\begin{equation*}
\begin{split}
\psi_x^i \circ (\mathcal Z^{(1)}_G, \mathcal Z^{(2)}_G ) = \mathcal Z^{(1)}_G
(&\psi_x^i, \psi_x^{2i}, \psi_x^{3i}, \dots, \\
&\psi_{x,y,y}^i, \psi_{x,y,y}^{2i}, \psi_{x,y,y}^{3i}, \dots)
\end{split}
\end{equation*}

\begin{equation*}
\begin{split}
\psi_{x,y,y}^i \circ (\mathcal Z^{(1)}_G, \mathcal Z^{(2)}_G ) = \Zfull_G
(&\psi_x^i, \psi_x^{2i}, \psi_x^{3i}, \dots, \\
&\psi_{x,y,y}^i, \psi_{x,y,y}^{2i}, \psi_{x,y,y}^{3i}, \dots)
\end{split}
\end{equation*}
\end{theorem}
Биструктуры подставляются вместо бицветов, моноструктуры, вместо моноцветов. В
остальном рассуждение дословно повторяет случай обычных species.

\begin{remark}
Если сделать подстановку 
$$
\psi_{x}^1 = t, \psi_{x}^k = 0, k>1
$$
$$
\psi_{x,y,y}^1 = s, \psi_{x,y,y}^k = 0, k>1
$$
То полученная формула показывает, что \ref{eq:comp} справедливо для
экспоненциальных производящих функций bilabeled-структур (то есть производящая
функция от двух переменных [TODO: А что такое \ref{eq:comp} в этом случае?]). А
можно сделать подстановку $s := t$, которая даст выполнение формулы
\ref{eq:comp} для exp-производящей функции просто labeled-структур [TODO: A
это разве что-то новое?].
\end{remark}

\subsubsection{Примеры цикленного индекса композиции}
\begin{example}
Посчитаем $(\mathcal Z^{(1)}, \mathcal Z^{(2)})(\dB \circ \dA)$
$$
\mathcal Z^{(1)}(\dB \circ \dA) = \psi_{x}^1 \circ \psi_{x, y}^1 = \psi_{x, y}^1
= \mathcal Z^{(1)}(\dA)
$$ 
$$
\mathcal Z^{(2)}(\dB \circ \dA) = \psi_{y}^1 \circ 0 = 0 = \mathcal Z^{(2)}(\dA)
$$
\end{example}
\begin{example}
Да и вобще, справедливо
$$
\mathcal Z^{(1)}(\dB \circ A) = \mathcal Z^{(1)}(A)
$$ 
$$
\mathcal Z^{(2)}(\dB \circ A) = \mathcal Z^{(2)}(A)
$$
А так же
$$
\mathcal Z^{(1)}(A \circ \dB) = \mathcal Z^{(1)}(A)
$$ 
$$
\mathcal Z^{(2)}(A \circ \dB) = \mathcal Z^{(2)}(A)
$$
\end{example}
Это дает некоторое понимание композиции. Так $A \circ \dB = \dB \circ A = A$.
То есть \dB является нейтральным элементом в монойде h-species по композиции.
Это несколько контр-интуитивно, поскольку в обычных species нейтральным
элементом являеться одноточечное множество. А его образом при вложении species в
h-species являеться \dA.

\begin{example}
Интересно посмотреть на композицию с \dA
\begin{equation}
\begin{split}
\Zfull(\dBB \circ \dA) = \frac{1}{2} (\frac{1}{2}(\psi_{x,y,y}^1 +
\psi_{x}^1))^2 + \frac{1}{2} (\frac{1}{2}(\psi_{x,y,y}^2 +
\psi_{x}^2)) = \\
\frac{1}{8}((\psi_{x}^1)^2 + (\psi_{x, y, y}^1)^2 +
2\psi_{x, y, y}^2 + 2(\psi_{x, y, y}^1\psi_{x}^1) + 2\psi_{x}^2) =
\Zfull(\dAA)
\end{split}
\end{equation}
\begin{remark}
Отcюда можно сделать предположение, что $\dBB \circ \dA = \dAA$.
То есть подстановка \dA~--- это <<стирание различий между противоположными
гранями>>.
\end{remark}
\end{example}
\begin{example}
Посчитаем для структуры $V$ <<вершина куба>>.
$$
\Zfull(V) = e^{\psi_{x,y,y}^{1} + \frac{\psi_{x,y,y}^{2}}{2!} +
\frac{\psi_{x,y,y}^{3}}{3!} + \dots} 
$$

$$
\mathcal Z^{(1)}(V) = e^{(\psi_{x}^{1} + \frac{\psi_{x}^{2}}{2!} +
\frac{\psi_{x}^{3}}{3!} + \dots) + (\psi_{y}^{2} + \frac{\psi_{y}^{4}}{2!} +
\frac{\psi_{y}^{6}}{3!} + \dots)} 
$$

Для структуры $H$ <<куб>>.
$$
\Zfull(H) = \mathcal Z^{(1)}(H) = e^{\psi_{x,y}^{1} +
\frac{\psi_{x,y}^{2}}{2!} + \frac{\psi_{x,y}^{3}}{3!} + \dots} 
$$
Нетрудно убедится что $\mathcal Z^{(i)}(V \circ \dA) = \mathcal Z^{(i)}(H)$.
\end{example}

\subsection{Цикленный индекс species, вложенных в h-species}
\begin{statement}
Пусть $G$ --- обычный species, вложенный в h-species. $\mathcal Z_G$ --- его
цикленный индекс.
$$(\mathcal Z^{(1)}_G, \mathcal Z^{(2)}_G)
(\psi_x^1, \psi_x^2, \psi_x^3, \dots, 
\psi_{x,y,y}^1, \psi_{x,y,y}^2, \psi_{x,y,y}^3, \dots)
 = (\mathcal Z_G(\psi_{x,y}^1, \psi_{x,y}^2, \psi_{x,y}^3, \dots), 0)$$
\end{statement}

\subsection{Применение цикленного индекса к решению задач о раскрасках}
\begin{problem}
Посчитать количество способов покрасить $n$-мерный куб в $k$ цветов с точностью
до изометрий. Иными словами, посчитать количество орбит при действии $B_n$ на
множестве всевозможно расскрашенных кубов. \url{http://math.stackexchange.com/questions/5697/coloring-the-faces-of-a-hypercube}.
\end{problem}
\begin{solution}
В нашей нотации это вопрос о количестве расскрасок пар граней в $k$ моноцветов и
$\frac{k(k-1)}{2}$ бицветов. Поскольку любая расскраска даст нам моноструктуру,
то производящая функция для количества расскрасок от размерности, будет равна
$\Zone_{\mathbb H}(kt, kt^2, kt^3, \dots, k^2t,
k^2t^2, k^2t^3, \dots) = exp(kt + kt^2 + kt^3 + \dots
+ \frac{k(k-1)}{2}t + \frac{k(k-1)}{2}t^2 + \frac{k(k-1)}{2}t^3 + \dots) =
exp(\frac{k(k+1)}{2}t + \frac{k(k+1)}{2}t^2 + \frac{k(k+1)}{2}t^3 + \dots) =
(exp(log(\frac{1}{1-t})))^{\frac{k(k+1)}{2}} = \frac{1}{1-t}^{\frac{k(k+1)}{2}}$
\end{solution}

\section{Заключение?}

        %\include{diploma-ch02}
        %\include{diploma-ch03}
        %\include{diploma-conclusion}
        %\include{diploma-safety}
\end{onehalfspacing}

% ------------------------------------------------------------------------------
% Список литературы
\phantomsection
\renewcommand{\refname}{Список литературы}
\addcontentsline{toc}{section}{Список литературы}

\bibliographystyle{gost780u}
\bibliography{biblio/diploma}

\end{document}
