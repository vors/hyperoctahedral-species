\section{Гипероктаэдральные комбинаторные виды}
\subsection{Определение}
Рассмотрим категорию $\HSet$. В ней объекты это множества, снабженные
дополнительным действием --- инволюцией. А стрелки, это морфизмы, сохраняющие инволюцию. 
Рассмотрим категорию $\HB$ --- группоид конечных множеств с инволюциями без неподвижных точек.
Функтор $F:\HB \rightarrow \HSet$ --- гипероктаэдральный (или кубический)
комбинаторный вид. Мы будем так же для краткости употреблять термин
\emph{h-species}. Группоид $\HB$ эквивалентен группоиду, объекты которого $\Bar
n = \{-n, -n+1, \dots, -1, 1, 2, \dots, n-1, n\}$, инволюция - смена знака.
$\Bar n$ мы интрепретируем как грани куба, на которых действует
гипероктаэдральная группа $B_n$ --- группа движений n-мерного куба.
Эта же группа действует на множестве $F[\Bar n]$, которое мы
интрепретируем как множество структур на множестве граней куба. Действие
$B_n$ возникает из перестановок граней.

\begin{remark}
При работе со species, мы имели мощную комбинаторную интуицию, которая
мотивировала категорные конструкции. В случае h-species мы переносим категорные
конструкции species на новый контекст и пытаемся дать комбинаторную
интерпретацию получившимся результатам.
\end{remark}

\begin{example}
Вид $\mathbf H$ --- структура куб. Он сопоставляет $\Bar n$ одно множество.
Действие $B_n$ тривиально.
\end{example}
\begin{example}
$\dA$ --- неразличимая пара граней ($\mathbf H_1$). $\dB$ --- различимая пара
граней.
Оба они принимают значение $\emptyset$ на всем, кроме $\Bar 1$. Второе
соответствует действию $B_1$ на 2-х точечном множестве.
\end{example}
\begin{example}
$\dAA$ --- куб размерности 2 ($\mathbf H_2$). $\dBB$ ---
куб размерности 2 с различимыми противоположными гранями.  Второе
соответствует действию $B_2$ на 4-х точечном множестве.
\end{example}
\begin{example}
Структура $\dB \times \dB$. Это не то же самое что \dBB, поскольку это <<упорядоченная пара \dB>>.
\end{example}

\subsection{Вложение species в h-species}
Обычные комбинаторные виды можно <<вложить>> в гипероктаэдральные. Это вложение
задается функторами $I_1:\HB \rightarrow \B$ ($\pm i \mapsto i$) и $I_2:\Set
\rightarrow \HSet$ (к множеству добавляется тождественная инволюция).
В композиции с $F$ они дают $I_1 \circ F \circ I_2 : \HB \rightarrow \HSet$.
Комбинаторно: species рассмотреть как структуру не на точках, а на
парах (неразличимых) граней.

\subsection{Сложение и умножение h-species}
Сложение и умножение определяются полностью аналогично species. Они введены в
работе Бержерона \cite{BergH}.

\subsection{Аналитический функтор для h-species}
Хочется построить аналог аналитического функтора для h-species. Мы считаем, что
правильная версия гипероктаэдрального аналитического функтора действует из
$\HSet$ в $\HSet$. По аналогии с обычным случаем строим $\mathcal F$ как левое
расширение Кана:

\begin{tikzpicture}
\label{comm:h-an}
	\node (B) {$HB$};
	\node (S1) [below of=B] {$HSet$};
	\node (S2) [right of=B, node distance=3cm] {$HSet$};
	\draw [right hook->] (B) to node [swap] {$i$} (S1);
	\draw [->] (B) to node {$F$} (S2);
	\draw [->] (S1) to node [swap] {$\mathcal F$} (S2);
\end{tikzpicture}

Аргумент $\mathcal F$ будем называть множеством цветов (в действительности это
множество с инволюцией, подробнее в замечании \ref{rem:mono-bi}).

\begin{equation}
\label{eq:h-an}
	\mathcal F(A) = \sum\limits_n F[\Bar n] \times A^{\Bar n} / B_n
\end{equation}
[TODO:Здесь нужно добавить ссылку на общую конструкцию]

$A^{\Bar n}$ --- это отображения (раскраски), сохраняющие инволюцию. По
аналогии с обычными species, мы мыслим $\mathcal F(A)$, как множество
$F$-структур, раскрашеных в цвета из $A$.
\begin{remark}
\label{rem:mono-bi}
В новой ситуации множество цветов $A$ уже является не просто множеством, а
множеством с инволюцией. Пара элементов $(-i, i)$ отображается либо
в один и тот же элемент $(a, a)$ (который инволюцией переводится в себя), либо
в пару элементов $(b, \Bar b)$, сопряженных инволюцией. Будем называть первый
случай \emph{моноцветом}, второй --- \emph{бицветом}.
\end{remark}

На $F[\Bar n] \times A^{\Bar n}$ действует группа $B_n \times
\mathbb Z_2$, второй сомножитель соответствует инволюции на цветах. После
факторизации по $B_n$, мы получаем объект из $HSet$.


%\begin{definition}
%Будем называть элементы $(F[\Bar n] \times
%A^{\Bar n})$ $A$-крашенными $F$ структурами на гранях $n$-мерного куба. Таким
%образом правую часть \ref{eq:h-an} можно интерпретировать как всевозможные
%классы эквивалентности крашеных структур.
%\end{definition}

\subsection{Декатегорификация аналитического функтора} 
Можно действовать наивно: написать производящую функцию для числа раскрасок
(\ref{eq:fr}), по аналогии с классическим случаем. Такая формула
(\ref{eq:h-fr1}) рассматривалась (в контексте теории представлений группы $S_n \wr G$) в работе
\url{http://www.combinatorics.org/ojs/index.php/eljc/article/download/v11i1r56/pdf}
(см. также приложение B во втором анлгийском издании книги Макдональда
\cite{Mac2}). При таком подходе определяются операции сложения и умножения
по Коши для цикленных индексов. Но попытки определить
гипероктаэдральных плетизм оказываются безуспешны. Выяснилось, что правильный аналог цикленного
индекса должен помнить информацию о следующем свойстве раскрашенной структуры.

В качестве множества цветов рассмотрим счетное множество моноцветов $x_1,
x_2, x_3, \dots$ объединенное с счетным множеством бицветов $y_1, y_2, y_3, \dots$.

\begin{proposal}
Покрашенные структуры сами по себе можно рассматривать как моноцвет, либо
бицвет. Это по--прежнему определяется длинной орбиты инволюции на $A$ (уже
после факторизации по $B_n$). Будем разделять раскрашенные структуры на
\emph{моноструктуры} и \emph{биструктуры}. 
\end{proposal}

\begin{example}
Раскрашенная в бицвет \dA, это моноструктура. А раскрашенная в тот же бицвет
\dB, это биструктура.
\end{example}

\begin{proposal}
Гипероктаэдральный цикленный индекс (аналог \ref{eq:fr}) определим как пару
симметрических (от $\{x_i\}, \{y_j\}$) функций $(\Zone, \Ztwo)$. Коофициент при
мономе $x_{i_1} \dots x_{i_k} y_{j_1} \dots y_{j_l}$ в $\Zone$
равен количеству моноструктур c раскраской $\{x_{i_1}, \dots, x_{i_k}, y_{j_1},
\dots, y_{j_l}\}$. Коофициент при том же мономе в $\Ztwo$ равен количеству
биструктур с такой раскраской. 
\end{proposal}

\begin{example}
$[\Zone, \Ztwo](\dB) = [(x_1 + x_2 + x_3 + \dots), (y_1 + y_2 + y_3 + \dots)]$
\end{example}

\begin{statement}
Количество орбит под действием $B_n \times \mathbb Z_2$ соответствует $\mathcal
Z^{(1)} + \Ztwo$, а под действием только $B_n$ соответствует $\Zone + 2\mathcal
Z^{(2)}$. 
\end{statement}
\begin{proof}
В первом случае каждая моноструктура и биструктура будет посчитана 1 раз.
А во втором каждая биструктура будет посчитан два раза, т.к. действие инволюции
<<склеивающей>> две части биструктуры не учтено.
\end{proof}

\subsubsection{Подсчет цикленного индекса}
\begin{remark}
Циклы в каждом элементе $B_n$ бывают двух типов:
\emph{длинные} --- каждая грань входит в цикл вместе со своей противоположной
гранью. Пример длинного цикла: $$\dhoB$$
И \emph{короткие} --- пара граней
лежит в симметричных, различных циклах. Пример короткого цикла: $$\dhoA$$
\end{remark}

Введем обозначения. Пусть $\lambda, \mu$ --- разбиения. $|\lambda| + |\mu| = n$.
$\lambda$ --- цикленный тип коротких перестановок, $\mu$ --- цикленный тип длинных перестановок. 
$\sigma$ --- перестановка цикленного типа $(\lambda, \mu)$. 
$z_{\lambda \mu}$ --- индекс класса сопряженности $\sigma$.
$\chi$ --- характер (перестановочного) представления заданного $F$.


Посчитаем количество неподвижных точек для $B_n$.
\begin{statement}
Неподвижные раскрашенные структуры, это в точности
те, у которых длинный цикл покрашен в моноцвету, а пара симметричных коротких 
может быть покрашена либо в моноцвет, либо в бицвет. Причем для каждой пары
коротких циклов есть ровно $2$ способа их покрасить в выбранный бицвет (эти
два способа сопряжены инволюцией на цветах).
\end{statement}
Под покрашенным циклом мы подразумеваем покраску всех его элементов в этот цвет
(такая покрашенная структура будет неподвижна относительно действия этого
элемента $B_n$).

\begin{statement}
Справедлива формула:
\begin{equation}
\label{eq:h-fr1}
\begin{split}
\mathcal Z_F^{(1)} + 2\mathcal Z_F^{(2)} = 
\sum_{n}\frac{1}{2^{n}n!}\sum_{\sigma \in B_n}&\chi(\sigma)
\psi_{x, y, y}^{\lambda(\sigma)} \psi_{x}^{\mu(\sigma)} = \\
\sum_{n, \lambda + \mu \vdash n}&\chi(\sigma_{\lambda \mu})
\frac{\psi_{x, y, y}^{\lambda} \psi_{x}^{\mu}}{z_{\lambda \mu}}
\end{split}
\end{equation}
\end{statement}
Здесь нижний индекс $\psi$ означает переменные по которым берется степенная
сумма. Например $\psi_{x, y, y}^2 =  (x_1^2 + x_2^2 + x_3^2 + \dots + y_1^2 +
y_2^2 + y_3^2 + \dots + y_1^2 + y_2^2 + y_3^2 + \dots) = (x_1^2 + x_2^2 + x_3^2
+ \dots + 2y_1^2 + 2y_2^2 + 2y_3^2 + \dots)$. При этом удвоенное $y$ в
$\psi_{x,y,y}$ (то есть коофициент $2$ у $y_i^k$) отражает тот факт, что можно
красить в бицвет 2-мя способами.

Посчитаем количество неподвижных точек для $B_n \times Z_2$. Разобъем
сумму на две части --- $(h, \Bar 0)$ и $(h, \Bar 1)$. Для первой формула будет
аналогична \ref{eq:h-fr1}, только из-за того что порядок группы в 2 раза больше,
появится коофициент $\frac{1}{2}$. 
Во второй части по-прежнему можно красить и длинные и короткие циклы в моноцвет.
А вот с бицветом происходит любопытная вещь --- предположим мы красим в него
цикл (пару циклов, в случае короткого). Тогда реальный цикл от $(h, \Bar 1)$
будет получатся из циклов $h$ добавлением <<смены грани>> на каждом шаге. 
Значит для циклов нечетной длинны сменится свойство короткий--длинный. Ниже
два поясняющих примера.
\begin{example}
Пусть перестановка $h_e = \dheA$.
Тогда $(h_e, \Bar 1) = \dheB$
\end{example}
\begin{example}
Пусть перестановка $h_o = \dhoA$.
Тогда $(h_o, \Bar 1) = \dhoB$
\end{example}

\begin{statement}
Справедлива формула:
\begin{equation}
\label{eq:h-fr2}
\begin{split}
\mathcal Z_F^{(1)} + \mathcal Z_F^{(2)} = 
\frac{1}{2}&
\sum_{n, \lambda + \mu \vdash n}\chi(\sigma_{\lambda \mu})
\frac{\psi_{x, y, y}^{\lambda} \psi_{x}^{\mu}}{z_{\lambda \mu}}
+ \\
\frac{1}{2}&
\sum_{n, \lambda_o + \mu_o + \lambda_e + \mu_e \vdash
n}\chi(\sigma_{\lambda_o \mu_o \lambda_e \mu_e})
\frac{\psi_{x, y, y}^{\lambda_e + \mu_o} \psi_{x}^{\mu_e + 
\lambda_o}}{z_{\lambda_o \mu_o \lambda_e \mu_e}}
\end{split}
\end{equation}
\end{statement}
Где $\lambda_o, \mu_o$ --- разбиения соответствующие коротким и длинным циклам
нечетной длинны, $\lambda_e, \mu_e$ --- четной.

Из формул \ref{eq:h-fr1}, \ref{eq:h-fr2} легко получить
\begin{equation}
\mathcal Z_F^{(1)} = 
\sum_{n, \lambda_o + \mu_o + \lambda_e + \mu_e \vdash
n}\chi(\sigma_{\lambda_o \mu_o \lambda_e \mu_e})
\frac{\psi_{x, y, y}^{\lambda_e + \mu_o} \psi_{x}^{\mu_e + 
\lambda_o}}{z_{\lambda_o \mu_o \lambda_e \mu_e}}
\end{equation}

\begin{equation}
\begin{split}
\mathcal Z_F^{(2)} = 
\frac{1}{2}&
\sum_{n, \lambda + \mu \vdash n}\chi(\sigma_{\lambda \mu})
\frac{\psi_{x, y, y}^{\lambda} \psi_{x}^{\mu}}{z_{\lambda \mu}}
- \\
\frac{1}{2}&
\sum_{n, \lambda_o + \mu_o + \lambda_e + \mu_e \vdash
n}\chi(\sigma_{\lambda_o \mu_o \lambda_e \mu_e})
\frac{\psi_{x, y, y}^{\lambda_e + \mu_o} \psi_{x}^{\mu_e + 
\lambda_o}}{z_{\lambda_o \mu_o \lambda_e \mu_e}}
\end{split}
\end{equation}

\subsubsection{Примеры вычисления цикленного индекса}
Посчитаем цикленные индексы для простых h-species.
Здесь мы будем писать $Z(A)$ вместо $Z_A$. Это не должно вызывать путаницу,
поскольку вместо $A$ будут использоваться схематические картинки. Их
никак не перепутать с переменными, от которых считается цикленный индекс. 

\begin{remark}
Формулы \ref{eq:h-fr1} и \ref{eq:h-fr2} подсказывают, что в практических
вычислениях в качестве симметричного базиса можно брать не $\{\psi_x^i,
\psi_y^j\}$ а $\{\psi_x^i, \psi_{x,y,y}^j\}$. 
Или другую линейную комбинацию, например $\{\psi_x^i, \psi_{x,y}^j\}$.
\end{remark}

\begin{example}
$$
\Zfull(\dA) = \frac{1}{2}(\psi_{x,y,y}^1 + \psi_{x}^1) = \psi_{x,y}^1
$$
$$
\Zone(\dA) = \frac{1}{2}(\psi_{x}^1 + \psi_{x, y, y}^1) = \psi_{x,y}^1
$$
Значит
$$
\Ztwo(\dA) = 0
$$
\end{example}
\begin{example}
$$
\Zfull(\dB) = \frac{1}{2}(2\psi_{x,y,y}^1 + 0\psi_{x}^1) = \psi_{x,y,y}^1
$$
$$
\Zone(\dB) = \frac{1}{2}(2\psi_{x}^1 + 0\psi_{x, y, y}^1) = \psi_{x}^1
$$
Значит
$$
\Ztwo(\dB) = \psi_{y}^1
$$
\end{example}
\begin{example}
$$
\Zfull(\dAA) = \frac{1}{8}((\psi_{x,y,y}^1)^2 + (\psi_{x}^1)^2 + 2\psi_{x}^2 +
2(\psi_x^1\psi_{x,y,y}^1) + 2\psi_{x,y,y}^2)
$$
Здесь коофициенты --- не характеры (характер при каждом слагаемом $= 1$).
$$
\Zone(\dAA) = \frac{1}{8}((\psi_{x}^1)^2 + (\psi_{x, y, y}^1)^2 +
2\psi_{x, y, y}^2 + 2(\psi_{x, y, y}^1\psi_{x}^1) + 2\psi_{x}^2) = \Zfull(\dAA)
$$
Последнее следовало и из общих соображений: легко видеть что $\mathcal
Z^{(2)}(\dAA) = 0$.
\end{example}
\begin{example}
$$
\Zfull(\dBB) = \frac{1}{8}(4(\psi_{x,y,y}^1)^2 + 0(\psi_{x}^1)^2 + 0\psi_{x}^2
+ 0(\psi_x^1\psi_{x,y,y}^1) + 2\times2\psi_{x,y,y}^2)
$$
$$
\Zone(\dBB) = \frac{1}{8}(4(\psi_{x}^1)^2 + 2\times2\psi_{x,y,y}^2)
$$
Откуда
$$
\Ztwo(\dBB) = \frac{1}{2}(\Zfull(\dBB) - \mathcal
Z^{(1)}(\dBB)) = \frac{1}{2}(\psi_{y,y}^1\psi_{x}^1 +
\frac{1}{2}(\psi_{y, y}^1)^2) = \psi_{y}^1\psi_{x}^1 + (\psi_{y}^1)^2 $$
\end{example}

\subsection{Сумма и произведение цикленных индексов}
\subsubsection{Сумма}
Сумма цикленных индексов соответсвует поточечной сумме аналитических
функторов и здесь нет никаких сюрпризов:
$$
\mathcal Z_{A + B}^{(1)} = \mathcal Z_A^{(1)} + \mathcal Z_B^{(1)}
$$
$$
\mathcal Z_{A + B}^{(2)} = \mathcal Z_A^{(2)} + \mathcal Z_B^{(2)}
$$
\subsubsection{Произведение}
Для произведения уже не совсем так. 
\begin{statement}
Моноструктура получается
в произведении двух моноструктур. А биструктура получается, если один из
сомножителей биструктура. Причем в случае, когда оба сомножителя ---
биструктуры, получается две различных биструктуры. 
\end{statement}
То есть
$$
\mathcal Z_{A * B}^{(1)} = \mathcal Z_A^{(1)} * \mathcal Z_B^{(1)}
$$
$$
\mathcal Z_{A * B}^{(2)} = 
\mathcal Z_A^{(1)} * \mathcal Z_B^{(2)} + 
\mathcal Z_A^{(2)} * \mathcal Z_B^{(1)} +
2 (\mathcal Z_A^{(2)} * \mathcal Z_B^{(2)})
$$
Откуда следует
$$
(\mathcal Z_{A * B}^{(1)} + 2\mathcal Z_{A * B}^{(2)}) = 
(\mathcal Z_A^{(1)} + 2\mathcal Z_A^{(2)}) * 
(\mathcal Z_B^{(1)} + 2\mathcal Z_B^{(2)})
$$

\begin{remark}
Это логично, поскольку $(\mathcal Z_F^{(1)} + 2\mathcal Z_F^{(2)})$ --- это
цикленный индекс для цветов, с <<забытой>> инволюцией.
\end{remark}

\subsubsection{Примеры цикленных индексов произведений}
Посчитаем произведение уже известных h-структур и их цикленных индексов.

\begin{example}
Структура $\dA \times \dA$.
$$
\Zone(\dA \times \dA) = \Zone(\dA) \times \mathcal
Z^{(1)}(\dA) = (\psi_{x, y}^1)^2
$$
\end{example}
\begin{example}
Структура $\dB \times \dB$.
$$
\Zfull(\dB \times \dB) = \Zfull(\dB) \times \Zfull(\dB) =(\psi_{x, y, y}^1)^2
$$
Легко получить эту же формулу и прямым подсчетом по формуле \ref{eq:h-fr1}, как
$\frac{1}{8}(8(\psi_{x, y, y}^1)^2)$.
$$
\Zone(\dB \times \dB) = \Zone(\dB) \times \mathcal
Z^{(1)}(\dB) =(\psi_{x}^1)^2
$$
\end{example}

\subsection{Цикленный индекс композиции}
Попробуем написать плетизм цикленных индексов для h-species,
по аналогии с обычными species.

\begin{problem}
Выразить
\begin{equation*}
\begin{split}
\Zi_{F \circ G} (&\psi_x^1, \psi_x^2, \psi_x^3, \dots, \\
						&\psi_y^1, \psi_y^2, \psi_y^3, \dots)
\end{split}
\end{equation*}
\end{problem}
Ответ будет дан в основной теореме \ref{th:z-main}. Рассмотрим доказательство
утверждения \ref{th:compos}. В нем мы <<красили>> каждую точку $F$-структуры в раскрашенную $G$-структуру. Сейчас у нас есть
бицвета и моноцвета. Исходя из определения, утверждаем, что
бицвета в цикленном индексе $\Zi_F$ нужно заменять на
цикленные индексы $\Ztwo_G$, отвечающие биструктурам $G$. А моноцвета заменяются
на цикленные индексы $\Zone_G$, отвечающие моноструктурам $G$. Так же как и в
случае обычных species, каждый цвет из $\Zi_G$ копируеться $k$ раз для $\psi^k$. 

\begin{theorem}[О гипероктаэдральном плетизме]
Справедливо следующее
\label{th:z-main}
\begin{equation}
\begin{split}
\label{eq:h-zfg}
	\Zi_{F \circ G} (\psi_x^1, \psi_x^2, \psi_x^3, &\dots, 
	\psi_y^1, \psi_y^2, \psi_y^3, \dots) = \\
	\mathcal Z_F^{(i)}(
		&\Zone_G(\psi_x^1, \psi_x^2, \psi_x^3, \dots, 
					 \psi_y^1, \psi_y^2, \psi_y^3, \dots), \\
		&\Zone_G(\psi_x^2, \psi_x^4, \psi_x^6, \dots, 
					 \psi_y^2, \psi_y^4, \psi_y^6, \dots), \\
		&\Zone_G(\psi_x^3, \psi_x^6, \psi_x^9, \dots, 
					 \psi_y^3, \psi_y^6, \psi_y^9, \dots), \\
		&\dots, \\
		&\Ztwo_G(\psi_x^1, \psi_x^2, \psi_x^3, \dots, 
					 \psi_y^1, \psi_y^2, \psi_y^3, \dots), \\
		&\Ztwo_G(\psi_x^2, \psi_x^4, \psi_x^6, \dots, 
					 \psi_y^2, \psi_y^4, \psi_y^6, \dots), \\
		&\Ztwo_G(\psi_x^3, \psi_x^6, \psi_x^9, \dots, 
					 \psi_y^3, \psi_y^6, \psi_y^9, \dots), \\
		&\dots
	)
\end{split}	
\end{equation}
\end{theorem}
Формула громоздкая, поэтому напишем ее на уровне членов:
\begin{equation*}
\begin{split}
\psi_x^i \circ (\Zone_G, \Ztwo_G ) = \Zone_G
(&\psi_x^i, \psi_x^{2i}, \psi_x^{3i}, \dots, \\
&\psi_y^i, \psi_y^{2i}, \psi_y^{3i}, \dots)
\end{split}
\end{equation*}

\begin{equation*}
\begin{split}
\psi_y^i \circ (\Zone_G, \Ztwo_G ) = \Ztwo_G
(&\psi_x^i, \psi_x^{2i}, \psi_x^{3i}, \dots, \\
&\psi_y^i, \psi_y^{2i}, \psi_y^{3i}, \dots)
\end{split}
\end{equation*}

\begin{remark}
Если сделать в \ref{eq:h-zfg} подстановку 
$$
\psi_{x}^1 = t, \psi_{x}^k = 0, k>1
$$
$$
\psi_y^1 = s, \psi_y^k = 0, k>1
$$
То получится формула
$$
\tilde{\mathcal Z}^{(i)}_{F \circ G} (t, s) = 
	\tilde{\mathcal Z}_F^{(i)} (
		\tilde{\mathcal Z}_G^{(1)} (t, s), 
		\tilde{\mathcal Z}_G^{(2)} (t, s))
$$
Таким образом аналог \ref{eq:comp} справедлив для
экспоненциальных производящих функций bilabeled-структур.
\end{remark}

\subsubsection{Примеры цикленного индекса композиции}
\begin{example}
Посчитаем $(\Zone, \Ztwo)(\dB \circ \dA)$
$$
\Zone(\dB \circ \dA) = \psi_{x}^1 \circ \psi_{x, y}^1 = \psi_{x, y}^1
= \Zone(\dA)
$$ 
$$
\Ztwo(\dB \circ \dA) = \psi_{y}^1 \circ 0 = 0 = \Ztwo(\dA)
$$
\end{example}
\begin{example}
Да и вобще, справедливо
$$
\Zi(\dB \circ A) = \Zi(A)
$$ 
$$
\Zi(A \circ \dB) = \Zi(A)
$$
\end{example}
\begin{remark}
Это дает некоторое комбинаторное понимание композиции. Так, видимо, $A \circ \dB
= \dB \circ A = A$. То есть \dB является нейтральным элементом в монойде h-species по
композиции.
Это несколько не интуитивно, поскольку в обычных species нейтральным
элементом являеться одноточечное множество. А его образом при вложении species в
h-species являеться \dA.
\end{remark}
\begin{example}
Интересно посмотреть на композицию с \dA
\begin{equation}
\begin{split}
\Zfull(\dBB \circ \dA) = \frac{1}{2} (\frac{1}{2}(\psi_{x,y,y}^1 +
\psi_{x}^1))^2 + \frac{1}{2} (\frac{1}{2}(\psi_{x,y,y}^2 +
\psi_{x}^2)) = \\
\frac{1}{8}((\psi_{x}^1)^2 + (\psi_{x, y, y}^1)^2 +
2\psi_{x, y, y}^2 + 2(\psi_{x, y, y}^1\psi_{x}^1) + 2\psi_{x}^2) =
\Zfull(\dAA)
\end{split}
\end{equation}
\begin{remark}
Отcюда можно сделать предположение, что $\dBB \circ \dA = \dAA$.
То есть подстановка \dA~--- это <<стирание различий между противоположными
гранями>>.
\end{remark}
\end{example}
\begin{example}
Посчитаем для структуры $V$ <<вершина куба>>.
$$
\Zfull(V) = e^{\psi_{x,y,y}^{1} + \frac{\psi_{x,y,y}^{2}}{2} +
\frac{\psi_{x,y,y}^{3}}{3} + \dots} 
$$
Эта формула получается по аналогии с $\mathcal Z_{\E}$. Коофициент при мономе
$x_1^{i_1}y_1^{j_1}\ldots$

$$
\Zone(V) = e^{(\psi_{x}^{1} + \frac{\psi_{x}^{2}}{2} +
\frac{\psi_{x}^{3}}{3} + \dots) + (\psi_{y}^{2} + \frac{\psi_{y}^{4}}{2} +
\frac{\psi_{y}^{6}}{3} + \dots)} 
$$

Для структуры $H$ <<куб>>.
$$
\Zfull(H) = \Zone(H) = e^{\psi_{x,y}^{1} +
\frac{\psi_{x,y}^{2}}{2!} + \frac{\psi_{x,y}^{3}}{3!} + \dots} 
$$
Тогда $\Zi(V \circ \dA) = \Zi(H)$, поскольку
при специализации всех $y$ в $0$, они равны.
\end{example}

\subsection{Цикленный индекс species, вложенных в h-species}
\begin{statement}
Пусть $G$ --- обычный species, вложенный в h-species. $\mathcal Z_G$ --- его
цикленный индекс.
$$(\Zone_G, \Ztwo_G)
(\psi_x^1, \psi_x^2, \psi_x^3, \dots, 
\psi_y^1, \psi_y^2, \psi_y^3, \dots)
 = (\mathcal Z_G(\psi_{x,y}^1, \psi_{x,y}^2, \psi_{x,y}^3, \dots), 0)$$
\end{statement}

\subsection{Применение цикленного индекса к решению задачи о раскрасках}
\begin{problem}
Посчитать количество способов покрасить $n$-мерный куб в $k$ цветов с точностью
до изометрий. Иными словами, посчитать количество орбит при действии $B_n$ на
множестве всевозможно раскрашенных кубов. \url{http://math.stackexchange.com/questions/5697/coloring-the-faces-of-a-hypercube}.
\end{problem}
\begin{solution}
В нашей нотации это вопрос о количестве раскрасок пар граней в $k$ моноцветов и
$\frac{k(k-1)}{2}$ бицветов. Поскольку любая раскраска даст нам моноструктуру,
то производящая функция для количества раскрасок от размерности, будет равна
$\Zone_{\mathbf H}(kt, kt^2, kt^3, \dots, k^2t,
k^2t^2, k^2t^3, \dots) = exp(kt + kt^2 + kt^3 + \dots
+ \frac{k(k-1)}{2}t + \frac{k(k-1)}{2}t^2 + \frac{k(k-1)}{2}t^3 + \dots) =
exp(\frac{k(k+1)}{2}t + \frac{k(k+1)}{2}t^2 + \frac{k(k+1)}{2}t^3 + \dots) =
(exp(log(\frac{1}{1-t})))^{\frac{k(k+1)}{2}} = \frac{1}{1-t}^{\frac{k(k+1)}{2}}$
\end{solution}
