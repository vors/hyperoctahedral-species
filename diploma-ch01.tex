\section{Гипероктаэдральные комбинаторные виды}
\subsection{Определение}
Рассмотрим категорию $\HSet$. В ней объекты это множества, снабженные
дополнительным действием --- инволюцией. А стрелки, это морфизмы, сохраняющие инволюцию. 
Рассмотрим категорию $\HB$ --- подкатегорию конечных множеств из
$\HSet$ с морфизмами только биекциями, и инволюциями без неподвижных точек.
Функтор $F:\HB \rightarrow \HSet$ --- гипероктаэдральный (или кубический)
это комбинаторный вид. Мы будем так же для краткости употреблять термин
\emph{h-species}. Объекты из $\HB$ будем отождествлять с $\Bar n = \{-n, -n+1,
\dots, -1, 1, 2, \dots, n-1, n\}$, где инволюция - смена знака. Эпитет
гипероктаэдральный используется потому, что на $\HB[\Bar n]$ действует
гипероктаэдральная группа $H_n$ --- группа движений n-мерного куба.
Некоторая не очень ясная комбинаторная интерпретация: множеству граней куба
сопоставляется множество структур на этих гранях, а действие $H_n$ возникает из
перестановок граней.

\begin{example}
Вид $\mathbb H$ --- структура куб. Он сопоставляет $\Bar n$ одно множество. Все
элементы $B_n$ переходят в тождественное отображение.
\end{example}
\begin{example}
$\dA$ --- неразличимая пара граней ($\mathbb H_1$). $\dB$ --- различимая пара
граней.
Оба они принимают значение $\emptyset$ на всем, кроме $\Bar 1$. Но во втором случае
$\dB(\Bar 1)$ это два объекта [TODO: правда??].
\end{example}
\begin{example}
Аналогично $\dAA$ --- структура куб размерности 2 ($\mathbb H_2$). $\dBB$ ---
куб размерности 2 с различимыми противоположными гранями. Их уже целых 4 [TODO:
правда??].
\end{example}

\subsection{Вложение species в h-species}
[TODO]

\subsection{Аналитический функтор для h-species}
Хочется построить аналог аналитического функтора для h-species

\begin{tikzpicture}
\label{comm:h-an}
	\node (B) {$HB$};
	\node (S1) [below of=B] {$HSet$};
	\node (S2) [right of=B, node distance=3cm] {$HSet$};
	\draw [right hook->] (B) to node [swap] {$i$} (S1);
	\draw [->] (B) to node {$F$} (S2);
	\draw [->] (S1) to node [swap] {$\mathcal F$} (S2);
\end{tikzpicture}

\begin{equation}
\label{eq:h-an}
	\mathcal F = \sum\limits_n F[\Bar n] \times A^{\Bar n} / B_n
\end{equation}
Где $A^{\Bar n}$ задает отображение, сохраняющее инволюцию. 

TODO:Здесь нужно добавить проверук универсальности картинки

\subsection{Декатегорификация аналитического функтора: цикленный индекс} 

\subsubsection{Случай h-species}
Попробуем построить аналогичную конструкцию для h-species.
Прежде всего отметим, что раскраска, элемент $A^{\Bar n}$, это отображение,
сохраняющее инволюцию. Значит элементы $n$ и $-n$ должны отображаться либо в
один и тот же элемент $A$ (который инволюцией переводиться в себя), либо в пару
элементов сопряженных инволюцией. Будем называть первый случай
\emph{моноцветом}, второй --- \emph{бицветом}.	

Покрашенные структуры сами по себе можно рассматривать как моноцвет, либо
бицвет. Это по--прежнему определяется длинной орбиты инволюции $A$, уже
после факторизации по $B_n$. То есть кроме действия $B_n$ есть еще внешняя
инволюция --- действие $Z_2$. Будем называть их \emph{моноструктурами} и
\emph{биструктурами}.

Цикленный индекс, считающий только моноструктуры будем обозначать
$\mathcal Z^{(1)}$, биструктуры --- $\mathcal Z^{(2)}$. Количество орбит под
действием $H_n \times Z_2$ соответствует $\mathcal Z^{(1)} + \mathcal Z^{(2)}$,
а под действием только $H_n$ соответствует $\mathcal Z^{(1)} + 2\mathcal
Z^{(2)}$. Поскольку каждая биструктура будет посчитан два раза.

В качестве H-множества цветов возьмем счетное множество моноцветов $x_1, x_2,
x_3, \dots$ объединенное с счетным множеством бицветов $y_1, y_2, y_3, \dots$.

Допустим, что мы придумали весовую функцию, отправляющую каждую расскрашенную
структуру в моном и любая орбита отправляеться в один моном. Применив Лемму
Бернсайда переходим к подсчету неподвижных точек. Циклы в каждом элементе $H_n$
бывают двух типов:
длинные --- каждая грань входит в цикл вместе со своей противоположной гранью и
короткие --- пара граней лежит в симметричных, различных циклах. 

Посчитаем количество количество неподвижных точек для $H_n$. Пусть $\lambda^1$
--- цикленный тип коротких перестановок, $\lambda^2$ --- цикленный тип длинных
перестановок. Утверждение: неподвижные раскрашенные структуры, это в точности
те, у которых длинный цикл соответсвует моноцвету, а пара симметричных коротких 
может быть покрашена либо в моноцвет, либо в бицвет.

Это можно выразить такой формулой:
\begin{equation}
\label{eq:h-fr1}
\begin{split}
\mathcal Z_F^{(1)} + 2\mathcal Z_F^{(2)} = 
\sum_{n}\frac{1}{2^{n}n!}\sum_{\sigma \in B_n}&\chi(\sigma)
\psi_{x, y, y}^{\lambda^1(\sigma)} \psi_{x}^{\lambda^2(\sigma)} = \\
\sum_{n, \lambda^1 + \lambda^2 \vdash n}&\chi(\sigma_{\lambda^1 \lambda^2})
\frac{\psi_{x, y, y}^{\lambda^1} \psi_{x}^{\lambda^2}}{z_{\lambda^1 \lambda^2}}
\end{split}
\end{equation}
Здесь нижний индекс $\psi$ означает переменные по которым берется степенная
сумма. Например $\psi_{x, y, y}^2 =  (x_1^2 + x_2^2 + x_3^2 + \dots + y_1^2 +
y_2^2 + y_3^2 + \dots + y_1^2 + y_2^2 + y_3^2 + \dots)$. При этом коофициент
2 у $y_i^2$ отражает тот факт, что можно раскрасить $k$ пар граней в бицвет,
так чтобы расскраска была неподвижна, под действием короткого цикла, 2-мя способами.

Посчитаем количество количество неподвижных точек для $H_n \times Z_2$. Разобъем
сумму на две части --- $(h, \Bar 0)$ и $(h, \Bar 1)$. Для первой формула будет
аналогична \ref{eq:h-fr1}, только из-за того что порядок группы в 2 раза больше,
появится коофициент $\frac{1}{2}$.

Во второй части по-прежнему можно красить и длинные и короткие циклы в моноцвет.
А вот с бицветом происходит любопытная вещь --- предположим мы красим цикл (пару
циклов в него). Тогда добавляется смена грани на каждом шаге, а значит для
циклов нечетной длинны сменится свойство короткий--длинный. Итоговая формула
такая 

\begin{equation}
\begin{split}
\mathcal Z_F^{(1)} + \mathcal Z_F^{(2)} = 
\frac{1}{2}&
\sum_{n, \lambda^1 + \lambda^2 \vdash n}\chi(\sigma_{\lambda^1 \lambda^2})
\frac{\psi_{x, y, y}^{\lambda^1} \psi_{x}^{\lambda^2}}{z_{\lambda^1 \lambda^2}}
+ \\
\frac{1}{2}&
\sum_{n, \lambda_o^1 + \lambda_o^2 + \lambda_e^1 + \lambda_e^2 \vdash
n}\chi(\sigma_{\lambda_o^1 \lambda_o^2 \lambda_e^1 \lambda_e^2})
\frac{\psi_{x, y, y}^{\lambda_e^1 + \lambda_o^2} \psi_{x}^{\lambda_e^2 + 
\lambda_o^1}}{z_{\lambda_o^1 \lambda_o^2 \lambda_e^1 \lambda_e^2}}
\end{split}
\end{equation}

Где $\lambda_o$ --- циклы нечетной длинны, $\lambda_e$ ---
циклы четной длинны.

Откуда легко получить
\begin{equation}
\label{eq:h-fr2}
\mathcal Z_F^{(1)} = 
\sum_{n, \lambda_o^1 + \lambda_o^2 + \lambda_e^1 + \lambda_e^2 \vdash
n}\chi(\sigma_{\lambda_o^1 \lambda_o^2 \lambda_e^1 \lambda_e^2})
\frac{\psi_{x, y, y}^{\lambda_e^1 + \lambda_o^2} \psi_{x}^{\lambda_e^2 + 
\lambda_o^1}}{z_{\lambda_o^1 \lambda_o^2 \lambda_e^1 \lambda_e^2}}
\end{equation}

\begin{equation}
\begin{split}
\mathcal Z_F^{(2)} = 
\frac{1}{2}&
\sum_{n, \lambda^1 + \lambda^2 \vdash n}\chi(\sigma_{\lambda^1 \lambda^2})
\frac{\psi_{x, y, y}^{\lambda^1} \psi_{x}^{\lambda^2}}{z_{\lambda^1 \lambda^2}}
- \\
\frac{1}{2}&
\sum_{n, \lambda_o^1 + \lambda_o^2 + \lambda_e^1 + \lambda_e^2 \vdash
n}\chi(\sigma_{\lambda_o^1 \lambda_o^2 \lambda_e^1 \lambda_e^2})
\frac{\psi_{x, y, y}^{\lambda_e^1 + \lambda_o^2} \psi_{x}^{\lambda_e^2 + 
\lambda_o^1}}{z_{\lambda_o^1 \lambda_o^2 \lambda_e^1 \lambda_e^2}}
\end{split}
\end{equation}

\subsubsection{Примеры}
Посчитаем цикленные индексы для простых h-species.
Здесь мы будем писать $Z(A)$ вместо $Z_A$. Это не должно вызывать путаницу,
поскольку вместо $A$ будут использоваться схематические картинки и их
не перепутать с переменными, от которых считаеться цикленный индекс. 

Структура <<одна пара граней>>, будем символически писать \dA.
$$
\Zfull(\dA) = \frac{1}{2}(\psi_{x,y,y}^1 + \psi_{x}^1) = \psi_{x,y}^1
$$
$$
\mathcal Z^{(1)}(\dA) = \frac{1}{2}(\psi_{x}^1 + \psi_{x, y, y}^1) = \psi_{x,y}^1
$$
Значит
$$
\mathcal Z^{(2)}(\dA) = 0
$$

Структура <<одна пара граней, грани различаются>>. Обозначение \dB.
$$
\Zfull(\dB) = \frac{1}{2}(2\psi_{x,y,y}^1 + 0\psi_{x}^1) = \psi_{x,y,y}^1
$$
$$
\mathcal Z^{(1)}(\dB) = \frac{1}{2}(2\psi_{x}^1 + 0\psi_{x, y, y}^1) = \psi_{x}^1
$$
Значит
$$
\mathcal Z^{(2)}(\dA) = \psi_{y}^1
$$

Структура <<квадрат>>. Обозначение \dAA.
$$
\Zfull(\dAA) = \frac{1}{8}((\psi_{x,y,y}^1)^2 + (\psi_{x}^1)^2 + 2\psi_{x}^2 +
2(\psi_x^1\psi_{x,y,y}^1) + 2\psi_{x,y,y}^2)
$$
Здесь коофициенты --- не характеры (характер при каждом слагаемом $= 1$).
$$
\mathcal Z^{(1)}(\dAA) = \frac{1}{8}((\psi_{x}^1)^2 + (\psi_{x, y, y}^1)^2 +
2\psi_{x, y, y}^2 + 2(\psi_{x, y, y}^1\psi_{x}^1) + 2\psi_{x}^2) = \Zfull(\dAA)
$$
Последнее следовало и из общих соображений: легко видеть что $\mathcal
Z^{(2)}(\dAA) = 0$.

Структура <<квадрат, противоположные грани различаются>>. Обозначение \dBB.
$$
\Zfull(\dBB) = \frac{1}{8}(4(\psi_{x,y,y}^1)^2 + 0(\psi_{x}^1)^2 + 0\psi_{x}^2
+ 0(\psi_x^1\psi_{x,y,y}^1) + 2\times2\psi_{x,y,y}^2)
$$
$$
\mathcal Z^{(1)}(\dBB) = \frac{1}{8}(4(\psi_{x}^1)^2 + 2\times2\psi_{x,y,y}^2)
$$
Откуда
$$
\mathcal Z^{(2)}(\dBB) = \frac{1}{2}(\Zfull(\dBB) - \mathcal
Z^{(1)}(\dBB)) = \frac{1}{2}(\psi_{y,y}^1\psi_{x}^1 +
\frac{1}{2}(\psi_{y, y}^1)^2) = \psi_{y}^1\psi_{x}^1 + (\psi_{y}^1)^2 $$

Структура $\dB \times \dB$. Это не то же самое что \dBB, поскольку это <<упорядоченная пара \dB>>.
Ее цикленный индекс мы посчитаем дальше.

\subsection{Сумма и произведение цикленных индексов}
\subsubsection{Сумма}
Сумма цикленных индексов соответсвует поточечной сумме аналитических
функторов и здесь нет никаких сюрпризов:
$$
\mathcal Z_{A + B}^{(1)} = \mathcal Z_A^{(1)} + \mathcal Z_B^{(1)}
$$
$$
\mathcal Z_{A + B}^{(2)} = \mathcal Z_A^{(2)} + \mathcal Z_B^{(2)}
$$
\subsection{Произведение}
Для произведения уже не совсем так. Утверждается, что моноструктура получается
в произведении двух моноструктур. А биструктура получается, если один из
сомножителей биструктура. Причем в случае, когда оба сомножителя ---
биструктуры, получается две различных биструктуры. То есть
$$
\mathcal Z_{A * B}^{(1)} = \mathcal Z_A^{(1)} * \mathcal Z_B^{(1)}
$$
$$
\mathcal Z_{A * B}^{(2)} = 
\mathcal Z_A^{(1)} * \mathcal Z_B^{(2)} + 
\mathcal Z_A^{(2)} * \mathcal Z_B^{(1)} +
2 (\mathcal Z_A^{(2)} * \mathcal Z_B^{(2)})
$$
Откуда следует
$$
(\mathcal Z_{A * B}^{(1)} + 2\mathcal Z_{A * B}^{(2)}) = 
(\mathcal Z_A^{(1)} + 2\mathcal Z_A^{(2)}) * 
(\mathcal Z_B^{(1)} + 2\mathcal Z_B^{(2)})
 $$
Что логично, поскольку $(\mathcal Z_F^{(1)} + 2\mathcal Z_F^{(2)})$ --- это
цикленный индекс для цветов, с <<забытой>> инволюцией.

\subsubsection{Примеры}
Посчитаем произведение уже известных h-структур и их цикленных индексов.

Структура $\dA \times \dA$.
$$
\mathcal Z^{(1)}(\dA \times \dA) = \mathcal Z^{(1)}(\dA) \times \mathcal
Z^{(1)}(\dA) = (\psi_{x, y}^1)^2
$$

Структура $\dB \times \dB$.
$$
\Zfull(\dB \times \dB) = \Zfull(\dB) \times \Zfull(\dB) =(\psi_{x, y, y}^1)^2
$$
Легко получить эту же формулу и из других соображений, как
$\frac{1}{8}(8(\psi_{x, y, y}^1)^2)$.
$$
\mathcal Z^{(1)}(\dB \times \dB) = \mathcal Z^{(1)}(\dB) \times \mathcal
Z^{(1)}(\dB) =(\psi_{x}^1)^2
$$

\subsection{Цикленный индекс композиции}
Теперь попробуем выстроить теорию композиции цикленного индекса для h-species,
параллельно теории species. Прежде всего отметим, что инволюция на множестве
цветов делит их на моноцвета ($x_1, x_2, x_3, \dots$) и бицвета ($y_1, y_2,
y_3, \dots$). Однако, формулы \ref{eq:h-fr1} и \ref{eq:h-fr2} подсказывают, что
в качестве базиса можно брать не $\psi_x^i, \psi_y^j$ а $\psi_x^i, \psi_{x,y,y}^j$. Впрочем это
тривиальная замена переменных.

Итак мы хотим выяснить чему равняются
\begin{equation*}
\begin{split}
\mathcal Z^{(1)}_{F \circ G} (&\psi_x^1, \psi_x^2, \psi_x^3, \dots, \\
						&\psi_{x,y,y}^1, \psi_{x,y,y}^2, \psi_{x,y,y}^3, \dots)
\end{split}
\end{equation*}

\begin{equation*}
\begin{split}
\mathcal Z^{(2)}_{F \circ G} (&\psi_x^1, \psi_x^2, \psi_x^3, \dots, \\
						&\psi_{x,y,y}^1, \psi_{x,y,y}^2, \psi_{x,y,y}^3, \dots)
\end{split}
\end{equation*}

Утверждается следующее:
\begin{equation}
\begin{split}
\label{eq:h-zfg}
	\mathcal Z^{(1)/(2)}_{F \circ G} (\psi_x^1, \psi_x^2, \psi_x^3, &\dots, 
	\psi_{x,y,y}^1, \psi_{x,y,y}^2, \psi_{x,y,y}^3, \dots) = \\
	\mathcal Z_F^{(1)/(2)}(
		&\mathcal Z^{(1)}_G(\psi_x^1, \psi_x^2, \psi_x^3, \dots, 
					 \psi_{x, y, y}^1, \psi_{x, y, y}^2, \psi_{x, y, y}^3, \dots), \\
		&\mathcal Z^{(1)}_G(\psi_x^2, \psi_x^4, \psi_x^6, \dots, 
					 \psi_{x, y, y}^2, \psi_{x, y, y}^4, \psi_{x, y, y}^6, \dots), \\
		&\mathcal Z^{(1)}_G(\psi_x^3, \psi_x^6, \psi_x^9, \dots, 
					 \psi_{x, y, y}^3, \psi_{x, y, y}^6, \psi_{x, y, y}^9, \dots), \\
		&\dots, \\
		&\Zfull_G(\psi_x^1, \psi_x^2, \psi_x^3, \dots, 
					 \psi_{x,y,y}^1, \psi_{x,y,y}^2, \psi_{x,y,y}^3, \dots), \\
		&\Zfull_G(\psi_x^2, \psi_x^4, \psi_x^6, \dots, 
					 \psi_{x,y,y}^2, \psi_{x,y,y}^4, \psi_{x,y,y}^6, \dots), \\
		&\Zfull_G(\psi_x^3, \psi_x^6, \psi_x^9, \dots, 
					 \psi_{x,y,y}^3, \psi_{x,y,y}^6, \psi_{x,y,y}^9, \dots), \\
		&\dots
	)
\end{split}	
\end{equation}
Эта формула слишком грамоздкая, поэтому давайте напишем ее на уровне членов:
\begin{equation*}
\begin{split}
\psi_x^i \circ (\mathcal Z^{(1)}_G, \mathcal Z^{(2)}_G ) = \mathcal Z^{(1)}_G
(&\psi_x^i, \psi_x^{2i}, \psi_x^{3i}, \dots, \\
&\psi_x^i, \psi_x^{2i}, \psi_x^{3i}, \dots)
\end{split}
\end{equation*}

\begin{equation*}
\begin{split}
\psi_{x,y,y}^i \circ (\mathcal Z^{(1)}_G, \mathcal Z^{(2)}_G ) = \Zfull_G
(&\psi_x^i, \psi_x^{2i}, \psi_x^{3i}, \dots, \\
&\psi_x^i, \psi_x^{2i}, \psi_x^{3i}, \dots)
\end{split}
\end{equation*}
Здесь мы пишем $(\mathcal Z^{(1)}_G, \mathcal Z^{(2)}_G )$, поскольку
цикленный индекс для h-species в действительности представляет собой пару.
Биструктуры подставляются вместо бицветов, моноструктуры, вместо моноцветов. В
остальном рассуждение дословно повторяет случай обычных species.

Аналогично, если сделать подстановку 
$$
\psi_{x}^1 = t, \psi_{x}^k = 0, k>1
$$
$$
\psi_{x,y,y}^1 = s, \psi_{x,y,y}^k = 0, k>1
$$

То полученная формула показывает, что \ref{eq:comp} справедливо для
экспоненциальных производящих функций bilabeled-структур (то есть производящая
функция от двух переменных). А можно сделать подстановку $s := t$, которая
даст выполнение формулы \ref{eq:comp} для exp-производящей функции просто
labeled-структур. А значит и обычной производящей функции unlabeled-структур.

\subsubsection{Примеры}
Посчитаем $(\mathcal Z^{(1)}, \mathcal Z^{(2)})(\dB \circ \dA)$
$$
\mathcal Z^{(1)}(\dB \circ \dA) = \psi_{x}^1 \circ \psi_{x, y}^1 = \psi_{x, y}^1
= \mathcal Z^{(1)}(\dA)
$$ 
$$
\mathcal Z^{(2)}(\dB \circ \dA) = \psi_{y}^1 \circ 0 = 0 = \mathcal Z^{(2)}(\dA)
$$
Да и вобще, справедливо
$$
\mathcal Z^{(1)}(\dB \circ A) = \mathcal Z^{(1)}(A)
$$ 
$$
\mathcal Z^{(2)}(\dB \circ A) = \mathcal Z^{(2)}(A)
$$
А так же
$$
\mathcal Z^{(1)}(A \circ \dB) = \mathcal Z^{(1)}(A)
$$ 
$$
\mathcal Z^{(2)}(A \circ \dB) = \mathcal Z^{(2)}(A)
$$
Это дает некоторое понимание композиции. Так $A \circ \dB = \dB \circ A = A$.
То есть \dB является нейтральным элементом в монойде h-species по композиции.
Это несколько контр-интуитивно, поскольку в обычных species нейтральным
элементом являеться одноточечное множество. А его образом при вложении species в
h-species являеться \dA. [TODO А не значит ли это что просто можно по другому вложить?]

Интересно посмотреть чем является \dA. Например
\begin{equation}
\begin{split}
\Zfull(\dBB \circ \dA) = \frac{1}{2} (\frac{1}{2}(\psi_{x,y,y}^1 +
\psi_{x}^1))^2 + \frac{1}{2} (\frac{1}{2}(\psi_{x,y,y}^2 +
\psi_{x}^2)) = \\
\frac{1}{8}((\psi_{x}^1)^2 + (\psi_{x, y, y}^1)^2 +
2\psi_{x, y, y}^2 + 2(\psi_{x, y, y}^1\psi_{x}^1) + 2\psi_{x}^2) =
\Zfull(\dAA)
\end{split}
\end{equation}
Откуда можно сделать вывод, что $\dBB \circ \dA = \dAA$.
То есть подстановка \dA, это <<стирание различий между противоположными
гранями>>.

\subsubsection{Предложения [TODO]}
$$(\mathcal Z^{(1)}_G, \mathcal Z^{(2)}_G)
(\psi_x^1, \psi_x^2, \psi_x^3, \dots, 
\psi_{x,y,y}^1, \psi_{x,y,y}^2, \psi_{x,y,y}^3, \dots)
 = (\mathcal Z_G(\psi_{x,y}^1, \psi_{x,y}^2, \psi_{x,y}^3, \dots), 0)$$

Где $G$ --- обычный species, вложенный в h-species. А $\mathcal Z_G$ --- его
цикленный индекс.
 
\subsection{Примеры [TODO]}
Посчитаем для структуры $V$ <<вершина куба>>.
$$
\Zfull(V) = e^{\psi_{x,y,y}^{1} + \frac{\psi_{x,y,y}^{2}}{2!} +
\frac{\psi_{x,y,y}^{3}}{3!} + \dots} 
$$

$$
\mathcal Z^{(1)}(V) = e^{(\psi_{x}^{1} + \frac{\psi_{x}^{2}}{2!} +
\frac{\psi_{x}^{3}}{3!} + \dots) + (\psi_{y}^{2} + \frac{\psi_{y}^{4}}{2!} +
\frac{\psi_{y}^{6}}{3!} + \dots)} 
$$

Для структуры куба $H$ <<куб>>.
$$
\Zfull(H) = \mathcal Z^{(1)}(H) = e^{\psi_{x,y}^{1} +
\frac{\psi_{x,y}^{2}}{2!} + \frac{\psi_{x,y}^{3}}{3!} + \dots} 
$$
Нетрудно убедится что $\mathcal Z^{(i)}(V \circ \dA) = \mathcal Z^{(i)}(H)$.
