\section{Формулы}
\subsection{Фробениусова характеристика}

В этом разделе мы напишем формулу для Фробениусовой характеристики.
То есть подчитаем количество неподвижных раскрасок.

Напомним, что в случае обычных species формула выглядит так:
\begin{equation}
\label{eq:fr}
\sum_{\lambda \vdash n}\chi(\sigma_{\lambda}) \frac{\phi^{\lambda}}{z_{\lambda}}
\end{equation}

Где $\chi$ --- характер (перестановочного) представления, $\sigma$ ---
перестановка цикленного типа $\lambda$, 
$\phi^{\lambda} = 
(x_1^{\lambda_1} + x_2^{\lambda_1} + x_3^{\lambda_1} + \dots)
(x_1^{\lambda_2} + x_2^{\lambda_2} + x_3^{\lambda_2} + \dots)
(x_1^{\lambda_3} + x_2^{\lambda_3} + x_3^{\lambda_3} + \dots)
\dots
$, $z_\lambda$ --- индекс класса сопряженности $\sigma$.
Появляется она из следующих соображений: в числителе стоит симметрическая
функция считающая все неподвижные раскраски. Цвета это $x_1, x_2, x_3, \dots$

Формула для h-species будет следующей
\begin{equation}
\label{eq:h-fr}
\sum_{\lambda_1 + \lambda_2 \vdash n}\chi(\sigma_{\lambda_1 \lambda_2})
\frac{(\phi^{\lambda_1} +
\bar{\phi^{\lambda_1}}) \phi^{\lambda_2}}{z_{\lambda}}
\end{equation}
%TODO: исправить черту над phi

